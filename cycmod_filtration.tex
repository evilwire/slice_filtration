\section{Filtration on Cycle Modules}


\begin{definition}
A \emph{homotopic module} $F_*$ is a $\Z$-graded homotopy 
invariant sheaf with transfers such that for every $n$, there
exists a map (called \emph{suspension}) $\susp_n: LF_n \to F_{n+1}$
such that the corresponding adjunction map $F_n \to RF_{n+1}$
is an isomorphism.

A map between homotopic modules $F_*$ and $G_*$ is a homomorphism 
of graded homotopy invariant sheaves with transfers $\phi_*$ that 
respects the suspension maps, i.e. the following diagram is 
commutative for each $n$:
\[
\begin{diagram}
LF_n          & \rTo{\susp_n}  & F_{n + 1}          \\
\dTo{L\phi_n} &                & \dTo_{\phi_{n + 1}} \\
LG_m          & \rTo{\susp'_n} & G_{m + 1}
\end{diagram}
\]
\end{definition}

The category $\HI_*$ is a well-powered, closed monoidal 
Grothendieck category. The main theorem in --, whose proof we 
reproduce here, is:

\begin{thm}\label{MCyc_HM_equiv}
There exists a categorical equivalence between $\CycMod$
and $\HI_*$.
\end{thm}
\begin{proof}(D\'eglise)
\end{proof}

Furthermore, as remarked in --, there exists adjoint functors
\[
\sptHI: \HI \leftrightarrows \HI_* : \loopHI
\]
in which $\sptHI$ is fully faithful, and $\loopHI$ is exact.

In the remainder of the section, we define a torsion filtration on
$HI_*$ which defines a functorial $\Z$-filtration. Through the 
categorical equivalence introduced in Theorem \ref{MCyc_HM_equiv}, 
$\CycMod$ admits the same structure. We proceed as we did for 
$\HI$, by introducing a sequence of coradicals (in this case, for 
each $n \in \Z$), and appeal to Theorem --.

Recall $\slHI{n}$ is the coradical associated to the counit
$L^{n + 1}R^{n + 1} \to id$. That is, for $F \in \HI$, $\slHI{n}F$
is the cokernel of $L^{n + 1}R^{n + 1}F \to F$.

\begin{lem}\label{R_commute_with_slHI}
For each $n \geq 0$, $R \slHI{n}$ is naturally isomorphic to 
$\slHI{n - 1}R$.
\end{lem}
\begin{proof}
Fix $F \in \HI$ and apply the exact functor $R$ to
\[
L^{n + 1}R^{n + 1} F \to F \to \slHI{n}F \to 0
\]
from which we obtain
\begin{equation}\label{R_counit_seq}
RL^{n + 1}R^{n + 1} \to RF \to R \slHI{n} F \to 0.
\end{equation}
But $RL^{n + 1}R^{n + 1}F \stackrel{\sim}{\to} L^nR^{n + 1} F$ is
an isomorphism and the map 
\[
L^nR^{n + 1}F \to RF
\] 
is the counit of $L^nR^n(RF) \to RF$.

Fitting (\ref{R_counit_seq}) with $L^nR^n(RF) \to RF \to \slHI{n 
- 1} \to 0$, we have
\begin{diagram}
L^{n + 1}R^{n + 1} F & \rTo & RF       & \rTo & R\slHI{n}F      & \rTo & 0 \\
\dTo{\sim}           &      & \dEquals &      & \dTo            \\
L^nR^n(RF)           & \rTo & RF       & \rTo & \slHI{n - 1} RF & \rTo & 0
\end{diagram}
Forcing $R\slHI{n}F \to \slHI{n - 1} RF$ to be an isomorphism.
\end{proof}

Fix $F_* \in \HI_*$ and $n \in \Z$, and let $\slHM{n}F_*$ 
represent the graded homotopy invariant sheaf with transfers 
whose $j$-th graded piece is given by
\[
(\slHM{n} F_*)_j = 
\begin{cases}
0                & \textrm{if } j + n < 0 \\
\slHI{n + j} F_j & \textrm{otherwise}
\end{cases}
\]
By Lemma \ref{R_commute_with_slHI}, we see that for $F_* \in 
\HI_*$, $\slHM{n}F_* \in \HI_*$. Functoriality is straightforward.
Furthermore, notice that for each $F_* \in \HI_*$, there exists a 
natural surjection (of total degree 0) $F_* \to \slHM{n} F_*$.

In particular,
\begin{lem}
For each $n \in \Z$, the functor $\slHM{n}$ is a coradical.
\end{lem}

\begin{thm}
There exists a $\Z$-filtration $\{\torTHM_n\}$ on $\HI_*$, which 
defines a functorial filtration. The torsion filtration on 
$\HI_*$ is compatible with the torsion filtration on $\HI$. 
Specifically, the following functor diagram commutes:
\begin{diagram}
\torT_n                 & \rTo{i_n} & \HI \\
\dTo{\sptHI|_{\torT_n}} &           & \dTo{\sptHI} \\
\torTHM_n               & \rTo{i_n} & \HI_*
\end{diagram}
\end{thm}
