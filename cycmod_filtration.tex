\section{Filtration on Cycle Modules}

\renewcommand{\Cat}[1]{\mathbf{#1}}

Let us recall definition and results of cycle modules, first
defined in \cite{Ro96}. In the following, let $S$ be a separated 
scheme of finite type over any base field $k$. Let $\fields(B)$ 
be the class of fields $F$ finitely generated over $k$ such that
$\spec F$ emits a map to $B$.

\begin{definition}
A \emph{cycle premodule} on $\fields(B)$ is an object function
that associated to each $F \in \fields(B)$ a $\Z$-graded abelian
group $M(F) = \prod_i M_i(K)$ with the following data:

\begin{enumerate}
\item[\textbf{D1.}] For each $\phi: F \to E$, there is a degree 0
map $\phi_*: M(F) \to M(E)$ called the \emph{restriction map 
associated to $\phi$}

\item[\textbf{D2.}] For each finite $\phi: F \to E$, there is a 
degree 0 map $\phi^*: M(E) \to M(F)$ called the \emph{corestriction
map associated to $\phi$}

\item[\textbf{D3.}] For each $F$, the group $M(F)$ is equipped
with a left $\milK_*(F)$-module, where $\milK_*(F)$ is the Milnor
$K$-ring of $F$.

\item[\textbf{D4.}] For any valuation $v$ of $F$, there exists 
maps $\res{v}: M(F) \to M(\resf(v))$ and $\special{v}{\idprime{p}}$
called the \emph{residue} and \emph{specialization} respectively, 
where $\resf(v)$ is the residue field of $v$ and $\idprime{p}$ is 
a prime of $v$,
\end{enumerate}

subject to the following conditions:

\begin{enumerate}
\item[\textbf{R1a.}] For each $\phi: F \to E$ and $\psi: E \to L$,
$(\psi \comp \phi)_* = \psi_* \comp \phi_*$

\item[\textbf{R1b.}] For each finite $\phi: F \to E$ and $\psi: E
\to L$, $(\psi \comp \phi)^* = \phi^* \comp \psi^*$

\item[\textbf{R1c.}] For $\phi: F \to E$ and $\psi: E \to L$ with
$\phi$ finite, define $R = E \otimes_F L$, and let $\idprime{p}$
be any prime ideal of $R$. (As $R$ is Artin, let $l_p$ be the 
length of the localized ring $R_{(\idprime{p})}$), and 
$\phi_{\idprime{p}}: L \to R/{\idprime{p}}$ and 
$\psi_{\idprime{p}}: E \to R/{\idprime{p}}$ be natural maps.
\[
\psi_*\comp \phi^* = \sum_{\idprime{p}} l_p \cdot 
(\phi_{\idprime{p}})^* \comp (\psi_{\idprime{p}})_*.
\]

\item[\textbf{R2.}] For $\phi: F \to E$, $x \in \milK_*F$, $y \in
\milK_*E$, $\rho \in M(F)$, and $\mu \in M(E)$, then:

\item[\textbf{R2a.}] $\phi_*(x \cdot \rho) = \phi_*(x) \cdot 
\phi_*(\rho)$.

\item[\textbf{R2b.}] if $\phi$ is finite, $\phi^*(\phi_*(x) \cdot 
\mu) = x \cdot \phi^*(\mu)$.

\item[\textbf{R2c.}] if $\phi$ is finite, $\phi^*(y \cdot 
\phi_*(\rho)) = \phi^*(y) \cdot \rho$.

\item[\textbf{R3.}] For $\phi: F \to E$, $v$ a valuation on $E$
and $w$ and a valuation on $F$:

\item[\textbf{R3a.}] Suppose $w$ is a nontrivial restriction of 
$v$ with ramification index $e$. Let $\phib: \resf{w} \to 
\resf{v}$ be the induced map. Then:
\[
\res{v} \comp \phi_* = e \phib_* \comp \res{w}.
\]

\item[\textbf{R3b.}] Let $\phi$ be finite. For each valuation $v$ 
an extension of $w$ to $E$, let $\phi_v: \resf{w} \to \resf{v}$
be the induced map. Then
\[
\res{v} \comp \phi^* \sum_{v} \comp \res{v}.
\]

\item[\textbf{R3c.}] Suppose $v$ restricts to a trivial valuation
$w$ on $F$. Then
\[
\res{v} \comp \phi_* = 0
\]

\item[\textbf{R3d.}] Suppose $v$ again restricts to a trivial 
valuation, and let $\phib: F \to \resf{v}$ be the induced map, and
$\idprime{p}$ a prime of $v$. Then
\[
\special{v}{\idprime{p}} \comp \phi_* = \phib_*
\]

\item[\textbf{R3e.}] Let $u$ be a $v$-unit, and $\rho \in M(F)$,
one has
\[
\res{v}(\{u\} \cdot \rho) = -\{\overline{u}\} \cdot \res{v}(\rho).
\]
\end{enumerate}
\end{definition}

\begin{definition}
A homomorphism $\omega: M \to M'$ of cycle premodules over 
$\fields(B)$ of even (resp. odd) type is given by homomorphisms:
\[
\omega_F: M(F) \to M'(F)
\]
such that for each $\phi: F \to E$
\begin{enumerate}
\item $\phi_* \comp \omega_F = \omega_E \comp \phi_*$

\item $\phi^* \comp \omega_E = \omega_F \comp \phi^*$

\item for $\{a\} \in \milK_*F$ and $\rho \in M(F)$, $\{a\} \cdot 
\omega_F(\rho) = \omega_F(\{a\} \cdot \rho)$ (resp. 
$\{a\} \cdot \omega_F(\rho) = -\omega_F(\{a\} \cdot \rho)$)

\item $\res{v} \comp \omega_F = \omega_{\resf{v}} \comp \res{v}$
(resp. $\res{v} \comp \omega_F = -\omega_{\resf{v}} \comp 
\res{v}$).
\end{enumerate}
\end{definition}

\begin{ex}
$\milK_*$ is a cycle premodule.
\end{ex}

For $X$ a $k$-scheme, let $\subsch{1}{X}$ denote the collection of 
codimension 1 subscheme. Write $\xi_X$ be the generic point of an
irreducible $X$ with $K_X = \O_X,\xi_X$. If $X$ is normal, then 
for $x \in \codim{1}{X}$, the local ring $\O_X,x$ is a valuation 
ring of $K_X$ with residue field $\resf{x}$. Write $M(x)$ for 
$M(\resf{x})$, and $\res{x}: M(\xi_X) \to M(x)$ for the 
restriction map.

Furthermore, for $x, y \in X$, let $Z$ be the closed subscheme 
determined by $x$, and $\overline{Z}$ be the normalization $Z$.
Define
\[
\ptres{x}{y}: M(x) \to M(y)
\]
by
\[
\ptres{x}{y} = 
\begin{cases}
0 & y \notin \subsch{1}{Z} \\
\sum_{z|y} \phi_{\resf{z},\resf{x}}^* \comp \res{z} & \textrm{otherwise}
\end{cases}
\]
In case $\ptres{x}{y}$ is nonzero, the sum is taken over all $z$
lying over $y \in \subsch{1}{Z}$, and
$\phi_{\resf{z},\resf{y}}^*$ is the corestriction map associated to 
the finite field extension $\resf{y} \to \resf{z}$.

\begin{definition}
A cycle module $M$ on $\fields(B)$ is a cycle premodule which
satisfies the following conditions:

\begin{enumerate}
\item[\textbf{(FD)}] \itemhead{Finite support of divisors.} 
$X$ be a normal scheme and $\rho \in M(\xi_X)$. Then $\res{x}: 
M(\xi_X) \to M(X)$ is 0 for all but finitely many $x \in 
\subsch{1}{X}$.

\item[\textbf{(C)}] \itemhead{Closedness.} If $X$ is an integral
local local of dimension 2 with closed point $x_0$, then the map 
from $M(\xi_X)$ to $M(x_0)$ given by
\[
\sum_{x \in \subsch{1}{X}} \ptres{x_0}{x} \comp \ptres{x}{\xi}
\]
is 0.
\end{enumerate}
\end{definition}

Let $\Cat{MCyc}$ denote the category of cycle modules. In ---,
D\'eglise demonstrated that $\Cat{MCyc}$ is categorically 
equivalent to the category of homotopic modules (see -- ), which
we represent by $\HI_*$. 

\begin{definition}
A \emph{homotopic module} $F_*$ is a $\Z$-graded homotopy 
invariant sheaf with transfers such that for every $n$, there
exists a map (called \emph{suspension}) $\susp_n: LF_n \to F_{n+1}$
such that the corresponding adjunction map $F_n \to RF_{n+1}$
is an isomorphism.

A map between homotopic modules $F_*$ and $G_*$ is a homomorphism 
of graded homotopy invariant sheaves with transfers $\phi_*$ that 
respects the suspension maps, i.e. the following diagram is 
commutative for each $n$:
\[
\begin{diagram}
LF_n          & \rTo{\susp_n}  & F_{n + 1}          \\
\dTo{L\phi_n} &                & \dTo_{\phi_{n + 1}} \\
LG_m          & \rTo{\susp'_n} & G_{m + 1}
\end{diagram}
\]
\end{definition}

The category $\HI_*$ is a well-powered, closed monoidal 
Grothendieck category. The main theorem in --, whose proof we 
reproduce here, is:

\begin{thm}\label{MCyc_HM_equiv}
There exists a categorical equivalence between $\Cat{MCyc}$
and $\HI_*$.
\end{thm}
\begin{proof}(D\'eglise)
\end{proof}

Furthermore, as remarked in --, there exists adjoint functors
\[
\sptHI: \HI \leftrightarrows \HI_* : \loopHI
\]
in which $\sptHI$ is fully faithful, and $\loopHI$ is exact.

In the remainder of the section, we define a torsion filtration on
$HI_*$ which defines a functorial $\Z$-filtration. Through the 
categorical equivalence introduced in Theorem \ref{MCyc_HM_equiv}, 
$\Cat{MCyc}$ admits the same structure. We proceed as we did for
$\HI$, by introducing a sequence of coradicals (in this case, for 
each $n \in \Z$), and appeal to Theorem --.

Recall $\slHI{n}$ is the coradical associated to the counit
$L^{n + 1}R^{n + 1} \to id$. That is, for $F \in \HI$, $\slHI{n}F$
is the cokernel of $L^{n + 1}R^{n + 1}F \to F$.

\begin{lem}\label{R_commute_with_slHI}
For each $n \geq 0$, $R \slHI{n}$ is naturally isomorphic to 
$\slHI{n - 1}R$.
\end{lem}
\begin{proof}
Fix $F \in \HI$ and apply the exact functor $R$ to
\[
L^{n + 1}R^{n + 1} F \to F \to \slHI{n}F \to 0
\]
from which we obtain
\begin{equation}\label{R_counit_seq}
RL^{n + 1}R^{n + 1} \to RF \to R \slHI{n} F \to 0.
\end{equation}
But $RL^{n + 1}R^{n + 1}F \stackrel{\sim}{\to} L^nR^{n + 1} F$ is
an isomorphism and the map 
\[
L^nR^{n + 1}F \to RF
\] 
is the counit of $L^nR^n(RF) \to RF$.

Fitting (\ref{R_counit_seq}) with $L^nR^n(RF) \to RF \to \slHI{n 
- 1} \to 0$, we have
\begin{diagram}
L^{n + 1}R^{n + 1} F & \rTo & RF       & \rTo & R\slHI{n}F      & \rTo & 0 \\
\dTo{\sim}           &      & \dEquals &      & \dTo            \\
L^nR^n(RF)           & \rTo & RF       & \rTo & \slHI{n - 1} RF & \rTo & 0
\end{diagram}
Forcing $R\slHI{n}F \to \slHI{n - 1} RF$ to be an isomorphism.
\end{proof}

Fix $F_* \in \HI_*$ and $n \in \Z$, and let $\slHM{n}F_*$ 
represent the graded homotopy invariant sheaf with transfers 
whose $j$-th graded piece is given by
\[
(\slHM{n} F_*)_j = 
\begin{cases}
0                & \textrm{if } j + n < 0 \\
\slHI{n + j} F_j & \textrm{otherwise}
\end{cases}
\]
By Lemma \ref{R_commute_with_slHI}, we see that for $F_* \in 
\HI_*$, $\slHM{n}F_* \in \HI_*$. Functoriality is straightforward.
Furthermore, notice that for each $F_* \in \HI_*$, there exists a 
natural surjection (of total degree 0) $F_* \to \slHM{n} F_*$.

In particular,
\begin{lem}
For each $n \in \Z$, the functor $\slHM{n}$ is a coradical.
\end{lem}

\begin{thm}
There exists a $\Z$-filtration $\{\torTHM_n\}$ on $\HI_*$, which 
defines a functorial filtration. The torsion filtration on 
$\HI_*$ is compatible with the torsion filtration on $\HI$. 
Specifically, the following functor diagram commutes:
\begin{diagram}
\torT_n                 & \rTo{i_n} & \HI \\
\dTo{\sptHI|_{\torT_n}} &           & \dTo{\sptHI} \\
\torTHM_n               & \rTo{i_n} & \HI_*
\end{diagram}
\end{thm}
