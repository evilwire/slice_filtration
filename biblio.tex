\bibliographystyle{plain}
\begin{thebibliography}{10}

\bibitem{BBD}
J. Bernstein, A. Beilinson and P. Deligne,
Faisceaux pervers, {\em Asterisque} 100 (1982).

\bibitem{BJV}
J. L. Bueso, P. Jara and A. Verschoren,
{\em Compatibility, Stability, and Sheaves}, 
Monographs and textbooks in pure and applied math.,
1995.

\bibitem{Deg08b}
F. D\'eglise, Motifs generiques.
{\em Rendiconti Sem. Mat. Univ. Padova},
119 (2008), 173 - 244.

\bibitem{Deg10}
F. D\'eglise, 
Modules homotopiques.
{\em Documenta Math.}
16 (2011), 411 - 455

\bibitem{HuKa}
F. Huber, B. Kahn, The slice filtration and mixed Tate motives,
Compositio Math. 142 (2006), 907 - 936.

\bibitem{MVW}
C. Mazza, V. Voevodsky, C. Weibel,
{\em Lecture notes on motivic cohomology},
Clay Mathematics Monographs vol. 2,
AMS, 2006

\bibitem{Ro96}
M. Rost, 
Chow groups with coefficients
{\em Documenta Math.} 
1 (1996), 319-393. 

\bibitem{Swan}
R. G. Swan,
Algebraic K-Theory
{\em Lecture Notes in Math},
Springer, 1968.

\bibitem{WH}
C. Weibel,
{\em An introduction to homological algebra},
Cambridge Univ. Press, 1994.

\end{thebibliography}
