\section{The Category of $\HI$}

\newcommand{\loopCM}{\epsilon}

In this section, we elaborate on the structure of the abelian
category of homotopy invariant sheaves with transfers ($\HI$) and 
show that $\HI$ has the following properties:

\begin{enumerate}

\item $\HI$ is a Grothendieck category that is well-powered.

\item $\HI$ is a closed monoidal category. That is, $\HI$ 
emits a symmetric tensor $\htensor$ and an internal hom
functor $\hhom$ such that
\[
\hom_{\HI}(F \htensor G, F') = \hom_{\HI}(F, \hhom(G, F'))
\]

We first define the tensor structure on $\HI$. By ---, there
is a $t$-structure on $\DMeff$, whose heart is $\HI$. Therefore, 
we may regard $\HI$ as a subcategory of $\DMeff$, whose 
reflection is the cohomology functor $\H^0$ associated to the
$t$-structure.
\end{enumerate}

\section{Homotopy Modules}

\newcommand{\cmcohom}{A}

Let $M$ be a cycle module, and write $M^p(X) = \oplus_{x \in 
\subsch{p}{X}} M(\resf{x})$. By ---, for all scheme $X$ 
essentially of finite type, the morphism
\[
d_M^p = \sum_{x \in \subsch{p}{X}, y \in \subsch{p + 1}{X}} 
\ptres{x}{y} : M^p(X) \to M^{p + 1}(X)
\]
is well defined, and fits into a chain complex
\[
\cdots \to M^p(X) \stackrel{d_M}{\to} M^{p + 1} \to \cdots
\]
Write $\cmcohom^p(X; M)$ for the $p$th cohomology group of the 
complex.

According to [Deg2.5], $\cmcohom{0}(-,M)$ defines a graded 
presheaf with transfers. According to [Deg06, 6.9], presheaf is 
in fact a sheaf. We write $F^M$ for the corresponding sheaf, and 
define a homotopic module strucutre as follows.

Let $X$ be a smooth scheme. Begin by considering the long exact
localization sequence (Deg.2.5c) associated to the zero section of
$X \to X \times \A^1$:
\[
0 \to F^M_n(X \times \A^1) \stackrel{j}{\to} F^M_n(\Gmpt \otimes X) 
   \stackrel{d}{\to} F^M_{n - 1}(X) \to \cdots
\]
where the map $j$ is given by the inclusion of $\Gmpt$ into $\A^1$,
and $d$ is the map induced by the degree $-1$ map
\[
\ptres{E}{0}: M(E(t)) \to M(E)
\]
associated to every valuation of $E(t)$ for every generic point 
$E$ of $X$.

Let $s_1: X \to \Gmpt \times X$ be the unit section. Recall that
$(F^M_n)_{-1}(X) = \ker (s_1^*)$. In fact, since $F^M_n$ is 
homotopy invariant, the canonical morphism $\ker s_1^* \to
\cok j^*$ is an isomorphism. Therefore, the map $\ptres{0}{X}$ 
induces a map
\[
\loopCM_n : (RF^M_{n})(X) \to F^M_{n - 1}(X).
\]
First, the localization sequence, introduced earlier, is 
compatible with the transfers on $X$. This follows from 2.5 and 
corollary 2.8. Therefore, $\loopCM_n$ defines is a map between 
homotopy invariant sheaves with transfers. Further, for all 
function field $E$, $\cmcohom(\A^1_E; M) = 0$ [Ros96, (2.2)H]. 
Thus, the fiber of $\loopCM_n$ in $E$ is an isomorphism, and by 
1.4, an isomorphism between homotopy invariant sheaves.

Therefore, $(F^M_*, \loopCM^{-1})$ defines a homotopic module that
which depends functorially on $M$.
