\section{(Co)torsion theory}

The language and methodology follows [BuJaVer]. For the 
following, let $\Cat{A}$ be an abelian category.

\begin{definition}
Let $F: \Cat{A} \to \Cat{A}$ be an endofunctor. We say that
$F$ is a \emph{subobject functor} if there exists a natural 
transformation $\tau$ from the identify functor to $F$
such that for each $X \to Y$, which gives rise to the following:
\[
\begin{diagram}
FX           & \rTo & FY           \\
\dTo{\tau_X} &      & \dTo{\tau_Y} \\
X            & \rTo & Y
\end{diagram}
\]
$\tau_X$ and $\tau_Y$ are injections 
\end{definition}

\begin{definition}
The concept of a quotient functor is dual to that of a suboject
functor. That is, $F: \Cat{A} \to \Cat{A}$ is a \emph{quotient
functor} if there exists a natural transformation $\tau$ from $F$ 
to the identity such that for each $X \to Y$, we have
\[
\begin{diagram}
X            & \rTo & Y            \\
\dTo{\tau_X} &      & \dTo{\tau_Y} \\
FX           & \rTo & FY
\end{diagram}
\]
and $\tau_X$ and $\tau_Y$ are surjections.
\end{definition}

\begin{definition}
We say that $F$ is \emph{idempotent} if $FF = F$.
\end{definition}

\begin{definition}
We say that a subobject functor $F$ is a \emph{pre-radical} (resp.
\emph{pre-coradical}) if for each $X \in \Cat{A}$, $F(X/F(X)) = 0$
(resp. $F(\ker X \to FX) = 0$).
\end{definition}

\begin{definition}
A pre-radical (resp. pre-coradical) $F$ is a \emph{radical} (resp. 
coradical if $F$ is left exact (resp. right exact).
\end{definition}

\begin{rmk}
Notice that being a quotient and being a subobject are dual 
notions. Similarly, (pre-)coradical is just the dual of radical. 
In the following, we consider only subobject functors, and 
(pre-)radical. Dual of the results whose arguments involve only 
``flipping the arrows'' will be stated without proof.
\end{rmk}

\begin{rmk}
In the case where $\Cat{A}$ is small, via Mitchell embedding 
there exists an embedding of $\Cat{A}$ as a subcategory of $R$-mod
for some suitable ring $R$. In this case, the subobject functors, 
quotient functors, idempotence, (pre-)radicals, and (pre-)coradicals
corresponds to their counterparts in the ring theoretic setting.
\end{rmk}

\begin{prop}
Any left exact subobject functor $F$ is idempotent. Dually, any
right exact quotient functor is idempotent.
\end{prop}

