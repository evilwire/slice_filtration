\section{Cycle Modules}

Let us recall definition and results of cycle modules, first
defined in \cite{Ro96}. In the following, let $S$ be a separated 
scheme of finite type over any base field $k$. Let $\fields(B)$ 
be the class of fields $F$ finitely generated over $k$ such that
$\spec F$ emits a map to $B$.

\begin{definition}
A \emph{cycle premodule} on $\fields(B)$ is an object function
that associated to each $F \in \fields(B)$ a $\Z$-graded abelian
group $M(F) = \prod_i M_i(K)$ with the following data:

\begin{enumerate}
\item[\textbf{D1.}] For each $\phi: F \to E$, there is a degree 0
map $\phi_*: M(F) \to M(E)$ called the \emph{restriction map 
associated to $\phi$}

\item[\textbf{D2.}] For each finite $\phi: F \to E$, there is a 
degree 0 map $\phi^*: M(E) \to M(F)$ called the \emph{corestriction
map associated to $\phi$}

\item[\textbf{D3.}] For each $F$, the group $M(F)$ is equipped
with a left $\milK_*(F)$-module, where $\milK_*(F)$ is the Milnor
$K$-ring of $F$.

\item[\textbf{D4.}] For any valuation $v$ of $F$, there exists 
maps $\res{v}: M(F) \to M(\resf(v))$ and $\Special{v}{\idprime{p}}$
called the \emph{residue} and \emph{specialization} respectively, 
where $\resf(v)$ is the residue field of $v$ and $\idprime{p}$ is 
a prime of $v$,
\end{enumerate}

subject to the following conditions:

\begin{enumerate}
\item[\textbf{R1a.}] For each $\phi: F \to E$ and $\psi: E \to L$,
$(\psi \comp \phi)_* = \psi_* \comp \phi_*$

\item[\textbf{R1b.}] For each finite $\phi: F \to E$ and $\psi: E
\to L$, $(\psi \comp \phi)^* = \phi^* \comp \psi^*$

\item[\textbf{R1c.}] For $\phi: F \to E$ and $\psi: E \to L$ with
$\phi$ finite, define $R = E \otimes_F L$, and let $\idprime{p}$
be any prime ideal of $R$. (As $R$ is Artin, let $l_p$ be the 
length of the localized ring $R_{(\idprime{p})}$), and 
$\phi_{\idprime{p}}: L \to R/{\idprime{p}}$ and 
$\psi_{\idprime{p}}: E \to R/{\idprime{p}}$ be natural maps.
\[
\psi_*\comp \phi^* = \sum_{\idprime{p}} l_p \cdot 
(\phi_{\idprime{p}})^* \comp (\psi_{\idprime{p}})_*.
\]

\item[\textbf{R2.}] For $\phi: F \to E$, $x \in \milK_*F$, $y \in
\milK_*E$, $\rho \in M(F)$, and $\mu \in M(E)$, then:

\item[\textbf{R2a.}] $\phi_*(x \cdot \rho) = \phi_*(x) \cdot 
\phi_*(\rho)$.

\item[\textbf{R2b.}] if $\phi$ is finite, $\phi^*(\phi_*(x) \cdot 
\mu) = x \cdot \phi^*(\mu)$.

\item[\textbf{R2c.}] if $\phi$ is finite, $\phi^*(y \cdot 
\phi_*(\rho)) = \phi^*(y) \cdot \rho$.

\item[\textbf{R3.}] For $\phi: F \to E$, $v$ a valuation on $E$
and $w$ and a valuation on $F$:

\item[\textbf{R3a.}] Suppose $w$ is a nontrivial restriction of 
$v$ with ramification index $e$. Let $\phib: \resf{w} \to 
\resf{v}$ be the induced map. Then:
\[
\res{v} \comp \phi_* = e \phib_* \comp \res{w}.
\]

\item[\textbf{R3b.}] Let $\phi$ be finite. For each valuation $v$ 
an extension of $w$ to $E$, let $\phi_v: \resf{w} \to \resf{v}$
be the induced map. Then
\[
\res{v} \comp \phi^* \sum_{v} \comp \res{v}.
\]

\item[\textbf{R3c.}] Suppose $v$ restricts to a trivial valuation
$w$ on $F$. Then
\[
\res{v} \comp \phi_* = 0
\]

\item[\textbf{R3d.}] Suppose $v$ again restricts to a trivial 
valuation, and let $\phib: F \to \resf{v}$ be the induced map, and
$\idprime{p}$ a prime of $v$. Then
\[
\Special{v}{\idprime{p}} \comp \phi_* = \phib_*
\]

\item[\textbf{R3e.}] Let $u$ be a $v$-unit, and $\rho \in M(F)$,
one has
\[
\res{v}(\{u\} \cdot \rho) = -\{\overline{u}\} \cdot \res{v}(\rho).
\]
\end{enumerate}
\end{definition}

\begin{definition}
A homomorphism $\omega: M \to M'$ of cycle premodules over 
$\fields(B)$ of even (resp. odd) type is given by homomorphisms:
\[
\omega_F: M(F) \to M'(F)
\]
such that for each $\phi: F \to E$
\begin{enumerate}
\item $\phi_* \comp \omega_F = \omega_E \comp \phi_*$

\item $\phi^* \comp \omega_E = \omega_F \comp \phi^*$

\item for $\{a\} \in \milK_*F$ and $\rho \in M(F)$, $\{a\} \cdot 
\omega_F(\rho) = \omega_F(\{a\} \cdot \rho)$ (resp. 
$\{a\} \cdot \omega_F(\rho) = -\omega_F(\{a\} \cdot \rho)$)

\item $\res{v} \comp \omega_F = \omega_{\resf{v}} \comp \res{v}$
(resp. $\res{v} \comp \omega_F = -\omega_{\resf{v}} \comp 
\res{v}$).
\end{enumerate}
\end{definition}

\begin{ex}
$\milK_*$ is a cycle premodule.
\end{ex}

For $X$ a $k$-scheme, let $\subsch{1}{X}$ denote the collection of 
codimension 1 subscheme. Write $\xi_X$ be the generic point of an
irreducible $X$ with $K_X = \O_X,\xi_X$. If $X$ is normal, then 
for $x \in \codim{1}{X}$, the local ring $\O_X,x$ is a valuation 
ring of $K_X$ with residue field $\resf{x}$. Write $M(x)$ for 
$M(\resf{x})$, and $\res{x}: M(\xi_X) \to M(x)$ for the 
restriction map.

Furthermore, for $x, y \in X$, let $Z$ be the closed subscheme 
determined by $x$, and $\overline{Z}$ be the normalization $Z$.
Define
\[
\ptres{x}{y}: M(x) \to M(y)
\]
by
\[
\ptres{x}{y} = 
\begin{cases}
0 & y \notin \subsch{1}{Z} \\
\sum_{z|y} \phi_{\resf{z},\resf{x}}^* \comp \res{z} & \textrm{otherwise}
\end{cases}
\]
In case $\ptres{x}{y}$ is nonzero, the sum is taken over all $z$
lying over $y \in \subsch{1}{Z}$, and
$\phi_{\resf{z},\resf{y}}^*$ is the corestriction map associated to 
the finite field extension $\resf{y} \to \resf{z}$.

\begin{definition}
A cycle module $M$ on $\fields(B)$ is a cycle premodule which
satisfies the following conditions:

\begin{enumerate}
\item[\textbf{(FD)}] \itemhead{Finite support of divisors.} 
$X$ be a normal scheme and $\rho \in M(\xi_X)$. Then $\res{x}: 
M(\xi_X) \to M(X)$ is 0 for all but finitely many $x \in 
\subsch{1}{X}$.

\item[\textbf{(C)}] \itemhead{Closedness.} If $X$ is an integral
local local of dimension 2 with closed point $x_0$, then the map 
from $M(\xi_X)$ to $M(x_0)$ given by
\[
\sum_{x \in \subsch{1}{X}} \ptres{x_0}{x} \comp \ptres{x}{\xi}
\]
is 0.
\end{enumerate}
\end{definition}

Let $\Cat{MCyc}$ denote the category of cycle modules. In ---,
D\'eglise demonstrated that $\Cat{MCyc}$ is categorically 
equivalent to the category of homotopic modules (see -- ), which
we represent by $\HI_*$. 

\vskip 10pt
The complex \[\gersseq{M}\]

We write $C_M^*$ for the Gersten complex associated to the cycle
module $M$. Rost showed that associated to every flat morphism 
$Y \to X$, there exists a contravariant morphism $C_M^*(X) \to 
C_M^*(Y)$, and to every equidimensional proper morphism $Y \to X$, 
a covariant $C_M^*(Y) \to C_M^*(X)$.

In [Deg06, 3.18], Deglise extended Rost's original work and 
associated to all local complete intersections $f: Y \to X$ for
which $Y$ emits resolution of singularities(see Deg06 3.13) a 
\emph{Gysin Map} 
\[
   f^*: C_M^*(X) \to C_M^*(Y)
\]
which is a map in the derived category of abelian groups, given 
by the composition of a morphism with the formal inverse of a 
quasi-isomorphism. More precisely, the latter is a formal inverse 
of the morphism $p^*$ for $p$ the projection map of a vector 
bundle. 

The Gysin map satisfies the following properties:
\begin{enumerate}
\item Since $f$ is also flat, $f^*$ coincide with the flat 
pullback map described earlier.

\item If $g: Z \to Y$ is a local complete intersection with $Z$
emitting resolution of singularities, then $(fg)^* = g^*f^*$.
\end{enumerate}

In the case where $f$ is a regular closed immersion, the 
hypothesis that $Y$ emits resolution of singularities is 
unnecessary. The map $f^*$ can be defined by using deformation
of the normal cone, following the original idea of Rost (cf 
[Deg.06, 3.3]).

We will use the following result due to Rost ([Ros96, 12.4]),
which partially describes the Gysin morphism:

\begin{prop}
Let $X$ be an integral scheme with function field $E$, and let $i: 
Z \to X$ be the closed immersion of a regular irreducible 
principal divisor, parametrized by $\idprime{p} \in \O_X(X)$. Let 
$v$ be the valuation of $E$ corresponding to the divisor $Z$. 
Therefore, the morphism $i^*: \cmcohom^0(X; M) \to \cmcohom^0(Z; 
M)$ is the restriction of $\Special{v}{\idprime{p}}: M(E) \to 
M(\resf{v})$.
\end{prop}

To each cartesian square
\[
\begin{diagram}
Y'      & \rTo{j} & X'      \\
\dTo{g} &         & \dTo{f} \\
Y       & \rTo{i} & X
\end{diagram}
\]
such that $i$ is a regular closed immersion, we associate a 
\emph{refined Gysin morphism} $\refgysin$ in the derived 
category:
\[
\refgysin: C_M^*(X') \to C_M^*(Y')
\]
that satisfies the following properties:
\begin{enumerate}
\item If $j$ is regular and the morphism of the normal cones
\[
N_{Y'}(X') \to g^{-1}N_Y(X)
\] 
is an isomorphism, then $\refgysin = j^*$.

\item If $f$ is proper, then $i^*f_* = g_*\refgysin$.
\end{enumerate}

Further, if the canonical immersion of the cone $C_{Y'}(X') \to
g^{-1}N_Y(X)$ in the normal fiber of $i$ is of pure codimension
equal to $e$, then the morphism $\refgysin$ is of cohomological
degree $e$.

For all smooth couples $(X, Y)$ and for all finite correspondence
$\alpha \in \Cor(X, Y)$, we define a map in the derived category:
\[
\alpha^* : C_M^*(Y) \to C_M^*(X)
\]
(see Deg06, 6.9). We describe this map as follows. Suppose that 
$\alpha$ is the class of a closed irreducible subscheme $Z$ of $X 
\times Y$. Consider the morphisms:
\[
X \stackrel{p}{\rightarrow} Z \stackrel{i}{\to} Z \times X \times Y
\stackrel{q}{\to} Y
\]
where $p$ and $q$ denote the canonical projections and $i$, the
graph of the closed immesion $Z \to X \times Y$. Therefore,
\begin{equation}
\alpha = p_* i^*q^*
\end{equation}
where $i^*$ denotes the Gysin map of a regular closed immersion 
$i$, $q^*$ the flat pullback and $p_*$ the finite push out.

The property $(\beta\alpha)^* = \alpha^*\beta^*$ is proved in 
[Deg06, 6.5].

\subsection{Localization Exact sequence}

We recall the localization exact sequence, following [Ros96],
and prove a supplementary result concerning its functoriality.
For a closed immersion $i: Z \to X$ purely of codimension $c$,
let $j: U \to X$ be the open immersion associated to its 
complement. By using functoriality mentioned above, we obtain
a short exact sequence of complexes:
\[
0 \to C_M^{p - c}(Z)_{n - c} \stackrel{i_*}{\to}
      C_M^p(X)_n \stackrel{j_*}{\to}
      C_M^p(U)_n \to 0.
\]
from which we have the following long exact sequence:
\[
\cdots \to \cmcohom^{p - c}_M(Z)_{n - c} \stackrel{i^*}{\to}
           \cmcohom^p_M(X)_n \stackrel{j^*}{\to}
           \cmcohom^p_M(U)_n \stackrel{\ptres{Z}{U}}{\to}
           \cmcohom^{p - c + 1}_M(Z)_{n - c} \to \cdots
\]
where the morphism $\ptres{Z}{U}$ is defined for the associated
graded by
\[
   \sum_{x \in \subsch{p}{U}, z \in \subsch{p - c + 1}{Z}} 
      \ptres{z}{x}.
\]
This sequence is natural with respect to proper and flat pushout.
We have the following proposition:

\begin{prop}
Consider the following cartesian square:
\[
\begin{diagram}
T       & \rTo{\iota'} & Z       \\
\dTo{k} &              & \dTo{i} \\
Y       & \rTo{\iota}  & X
\end{diagram}
\]
for which $\iota$ is a regular closed immersion. Suppose that $i$
(resp $k$) is a a closed immersion complementary an open 
immersion $j: U \to X$ (resp $l: V \to X$). Let $h: V \to U$ be
the map induced by $\iota$. Finally, suppose that $i$ (resp. $k$)
is of pure codimension equal to $c$ (resp. $d$). Then, the 
following diagram is commutative
\[
\begin{diagram}
\cdots & \rTo & \cmcohom^{p - c}_M(Z)_{n - c} & \rTo{i_*} & 
   \cmcohom^{p}_M(X)_n & \rTo{j^*} & \cmcohom^p_M(U) & 
   \rTo{\ptres{Z}{U}} & \cmcohom^{p - c + 1}_M(Z)_{n - c} & \rTo 
   & \cdots \\
       &      & \dTo{\refgysin}               &           & 
       \dTo{\iota^*}        &           & \dTo{h^*}       &                    
       & \dTo{\refgysin}                   \\
\cdots & \rTo & \cmcohom^{p - d}_M(T)_{n - d} & \rTo{i_*} & 
   \cmcohom^{p}_M(Y)_n & \rTo{j^*} & \cmcohom^p_M(V) & 
   \rTo{\ptres{T}{V}} & \cmcohom^{p - d + 1}_M(T)_{n - d} & \rTo 
   & \cdots
\end{diagram}
\]
\end{prop} 

\begin{rmk}
We could generalize the previous proposition to the case of 
refined Gysin maps as in Prop. 4.5 of [Deg06]. We leave this as a 
task to the reader.
\end{rmk}

\begin{rmk}
While the hypothesis of $i$ having pure codimension is natural, 
the one about $k$ is not, especially in the case where $Y$ does 
not intersect $Z$ transversely. Here, the hypothesis serves only
to determine the cohomological degree of the maps, and the 
hypothesis could easily be replaced by the assumption that the
maps are homogeneous with respect to cohomology degree.
\end{rmk}

\begin{proof}
It suffice to resume the proof of Proposition 4.5 of [Deg06]
for the following case:
\[
\begin{diagram}
T               & \rTo        &   & Z \\
\dTo(0,4)       & \rdTo{k}    &   & \dLine(0,4) &\rdTo{i}   \\
                &             & Y & \rTo        &           & X \\
                & \ldEquals   &   & \dTo        & \ldEquals \\
Y               & \rTo{\iota} &   & X
\end{diagram}
\]
In this case, we have the following commutative diagram, with
analogous notation as \emph{loc. cit.}
%\[
%\begin{diagram}
%\end{diagram}
%\]
The squares (1), (2), and (3) are commutative by similar reasons
as given in \emph{loc. cit.}, and (1'), (2') and (3') are 
clearly commutative. Each map that appear in the diagram are
well-defined maps of complexes, and induce maps between long exact
localization sequences.

To conclude, it suffices to point out that the maps $p^*$,
$p_T^*$ and $p_U^*$ are quasi-isomorphisms.
\end{proof}

\begin{cor}
Consider the following cartesian square:
\[
\begin{diagram}
T       & \rTo{g}      & Z       \\
\dTo{k} &              & \dTo{i} \\
Y       & \rTo{f}      & X
\end{diagram}
\]
of smooth schemes such that $i$ (resp. $k$) is a closed immersion 
of pure codimension equal to $c$, with a complement open immersion 
$j: U \to X$ (resp. $l: V \to X$). Let $h: V \to U$ represent the 
map induced by $f$. Then, the following diagram is commutative:
\[
\begin{diagram}
\cdots & \rTo & \cmcohom^{p - c}_M(Z)_{n - c} & \rTo{i_*} & 
   \cmcohom^{p}_M(X)_n & \rTo{j^*} & \cmcohom^p_M(U) & 
   \rTo{\ptres{Z}{U}} & \cmcohom^{p - c + 1}_M(Z)_{n - c} & \rTo 
   & \cdots \\
       &      & \dTo{g^*}                     &           
       & \dTo{f^*}           &           & \dTo{h^*}       
       &                    & \dTo{g^*}                         \\
\cdots & \rTo & \cmcohom^{p - c}_M(T)_{n - c} & \rTo{i_*} & 
   \cmcohom^{p}_M(Y)_n & \rTo{j^*} & \cmcohom^p_M(V) & 
   \rTo{\ptres{T}{V}} & \cmcohom^{p - c + 1}_M(T)_{n - c} & \rTo 
   & \cdots
\end{diagram}
\]
\end{cor}

\begin{rmk}
In [Deg08B], a closed pair is a couple $(X, Z)$ such that $X$ is a
smooth scheme and $Z$ is a closed subscheme. We say that $(X, Z)$ 
is \emph{smooth} (resp. \emph{of codimension $n$}) if $Z$ is 
smooth (resp.  purely of codimension $n$ in $X$).  

If $i: Z \to X$ is the associated closed immersion, a 
\emph{morphism of closed pairs} $(f, g)$ is a commutative square
\[
\begin{diagram}
T       & \rTo{g}      & Z       \\
\dTo{k} &              & \dTo{i} \\
Y       & \rTo{f}      & X
\end{diagram}
\]
which is topologically cartesian. We say that $(f, g)$ is 
\emph{cartesian} (resp. \emph{transverse}) when the square is
cartesian (resp. and the morphism induced on the normal cones
\[
C_T Y \to g^{-1}C_Z X
\]
is an isomorphism.)

The preceding corollary shows that the localization sequence 
associated to a cycle module $M$ and a closed pair $(X, Z)$ is 
natural with respect to transverse morphisms.
\end{rmk}

\subsection{Associated homotopy module}

Let $M$ be a cycle module, and write $M^p(X) = \oplus_{x \in 
\subsch{p}{X}} M(\resf{x})$. By ---, for all scheme $X$ 
essentially of finite type, the morphism
\[
d_M^p = \sum_{x \in \subsch{p}{X}, y \in \subsch{p + 1}{X}} 
\ptres{x}{y} : M^p(X) \to M^{p + 1}(X)
\]
is well defined, and fits into a chain complex
\[
\cdots \to M^p(X) \stackrel{d_M}{\to} M^{p + 1} \to \cdots
\]
Write $\cmcohom^p(X; M)$ for the $p$th cohomology group of the 
complex.

According to [Deg2.5], $\cmcohom{0}(-,M)$ defines a graded 
presheaf with transfers. According to [Deg06, 6.9], presheaf is 
in fact a sheaf. We write $F^M$ for the corresponding sheaf, and 
define a homotopic module strucutre as follows.

Let $X$ be a smooth scheme. Begin by considering the long exact
localization sequence (Deg.2.5c) associated to the zero section of
$X \to X \times \A^1$:
\[
0 \to F^M_n(X \times \A^1) \stackrel{j}{\to} F^M_n(\Gmpt \otimes X) 
   \stackrel{d}{\to} F^M_{n - 1}(X) \to \cdots
\]
where the map $j$ is given by the inclusion of $\Gmpt$ into $\A^1$,
and $d$ is the map induced by the degree $-1$ map
\[
\ptres{E}{0}: M(E(t)) \to M(E)
\]
associated to every valuation of $E(t)$ for every generic point 
$E$ of $X$.

Let $s_1: X \to \Gmpt \times X$ be the unit section. Recall that
$(F^M_n)_{-1}(X) = \ker (s_1^*)$. In fact, since $F^M_n$ is 
homotopy invariant, the canonical morphism $\ker s_1^* \to
\cok j^*$ is an isomorphism. Therefore, the map $\ptres{0}{X}$ 
induces a map
\[
\loopCM_n : (RF^M_{n})(X) \to F^M_{n - 1}(X).
\]
First, the localization sequence, introduced earlier, is 
compatible with the transfers on $X$. This follows from 2.5 and 
corollary 2.8. Therefore, $\loopCM_n$ defines is a map between 
homotopy invariant sheaves with transfers. Further, for all 
function field $E$, $\cmcohom(\A^1_E; M) = 0$ [Ros96, (2.2)H]. 
Thus, the fiber of $\loopCM_n$ in $E$ is an isomorphism, and by 
1.4, an isomorphism between homotopy invariant sheaves.

Therefore, $(F^M_*, \loopCM^{-1})$ defines a homotopic module that
which depends functorially on $M$.

\section{Equivalence of Categories between Homotopic Modules and 
Cycle Modules}

\subsection{Generic transformations} Consider a couple $(E, n)$
of a function field $E$ and an integer $n$. Recall that in 
[Deg08B 3.3.1] we have associated to $(E, n)$ a \emph{generic 
motive}
\[
M(E)\{n\} = ``\varprojlim_{A \subset E}'' M(\spec(A))\{n\}
\]
in the category of pro-objcts of $\DM_{gm}(k)$. We write 
$\GenMot$ for the category of generic motives.

Let $(F_*, \susp_n)$ be a homotopic module. Recall from -- that
$(F_*, \susp_n)$ can be identified with a functor $\phi: \DM_{gm}
\to \Ab$, and let $\widehat{\phi}$ be its canonical extension to
the category of pro-$\DM_{gm}$. As established in [Deg08B, 6.2.1],
the restriction of $\widehat{\phi}$ to $\GenMot$ is a cycle module
that we write as $\widehat{F}_*$, and that we call the generic
transformation of $F_*$.

We reiterate some key parts of its construction from [Deg08B]. 
First, note that for all generic motif $M(E)\{n\}$, 
$\widehat{\phi}(M(E)\{n\}) = \widehat{F}_{-n}(E)$ is nothing
but the fiber of $F_{-n}$ at $E$ (cf. 1.4). We can also interpret
the generic transformation $\widehat{F}$ as a system of fibers 
of $F$, together with specialization morphisms between them,
which satisfy the following structure axioms of cycle premodules:

\begin{enumerate}
\item[\textbf{D1.}] Functoriality (which is clear).

\item[\textbf{D2.}] Following [Deg08B 5.2], for a finite extension
$E \to L$, let $X$ and $Y$ be models for $E$ and $L$ respectively.
It follows that there exists a finite surjective morphism $f: Y 
\to X$ for which the induced map on the function fields is 
precisely that of $E \to L$.

The graph of $f$ regarded as a cycle of $X \times Y$ define a 
finite correspondence from $X$ to $Y$. This in turn yields a map 
$F_*(X) \to F_*(Y)$. We can show that this map is compatible with 
the restriction to an open subscheme of $X$, and induces the 
desired functoriality.

\item[\textbf{D3.}] 

\item[\textbf{D4.}]
\end{enumerate}
