\section{Cycle Modules}

Let us recall definition and results of cycle modules, first
defined in \cite{Ro96}. In the following, let $S$ be a separated 
scheme of finite type over any base field $k$. Let $\fields(B)$ 
be the class of fields $F$ finitely generated over $k$ such that
$\spec F$ emits a map to $B$.

\begin{definition}
A \emph{cycle premodule} on $\fields(B)$ is an object function
that associated to each $F \in \fields(B)$ a $\Z$-graded abelian
group $M(F) = \prod_i M_i(K)$ with the following data:

\begin{enumerate}
\item[\textbf{D1.}] For each $\phi: F \to E$, there is a degree 0
map $\phi_*: M(F) \to M(E)$ called the \emph{restriction map 
associated to $\phi$}

\item[\textbf{D2.}] For each finite $\phi: F \to E$, there is a 
degree 0 map $\phi^*: M(E) \to M(F)$ called the \emph{corestriction
map associated to $\phi$}

\item[\textbf{D3.}] For each $F$, the group $M(F)$ is equipped
with a left $\milK_*(F)$-module, where $\milK_*(F)$ is the Milnor
$K$-ring of $F$.

\item[\textbf{D4.}] For any valuation $v$ of $F$, there exists 
maps $\res{v}: M(F) \to M(\resf(v))$ and $\Special{v}{\idprime{p}}$
called the \emph{residue} and \emph{specialization} respectively, 
where $\resf(v)$ is the residue field of $v$ and $\idprime{p}$ is 
a prime of $v$,
\end{enumerate}

subject to the following conditions:

\begin{enumerate}
\item[\textbf{R1a.}] For each $\phi: F \to E$ and $\psi: E \to L$,
$(\psi \comp \phi)_* = \psi_* \comp \phi_*$

\item[\textbf{R1b.}] For each finite $\phi: F \to E$ and $\psi: E
\to L$, $(\psi \comp \phi)^* = \phi^* \comp \psi^*$

\item[\textbf{R1c.}] For $\phi: F \to E$ and $\psi: E \to L$ with
$\phi$ finite, define $R = E \otimes_F L$, and let $\idprime{p}$
be any prime ideal of $R$. (As $R$ is Artin, let $l_p$ be the 
length of the localized ring $R_{(\idprime{p})}$), and 
$\phi_{\idprime{p}}: L \to R/{\idprime{p}}$ and 
$\psi_{\idprime{p}}: E \to R/{\idprime{p}}$ be natural maps.
\[
\psi_*\comp \phi^* = \sum_{\idprime{p}} l_p \cdot 
(\phi_{\idprime{p}})^* \comp (\psi_{\idprime{p}})_*.
\]

\item[\textbf{R2.}] For $\phi: F \to E$, $x \in \milK_*F$, $y \in
\milK_*E$, $\rho \in M(F)$, and $\mu \in M(E)$, then:

\item[\textbf{R2a.}] $\phi_*(x \cdot \rho) = \phi_*(x) \cdot 
\phi_*(\rho)$.

\item[\textbf{R2b.}] if $\phi$ is finite, $\phi^*(\phi_*(x) \cdot 
\mu) = x \cdot \phi^*(\mu)$.

\item[\textbf{R2c.}] if $\phi$ is finite, $\phi^*(y \cdot 
\phi_*(\rho)) = \phi^*(y) \cdot \rho$.

\item[\textbf{R3.}] For $\phi: F \to E$, $v$ a valuation on $E$
and $w$ and a valuation on $F$:

\item[\textbf{R3a.}] Suppose $w$ is a nontrivial restriction of 
$v$ with ramification index $e$. Let $\phib: \resf{w} \to 
\resf{v}$ be the induced map. Then:
\[
\res{v} \comp \phi_* = e \phib_* \comp \res{w}.
\]

\item[\textbf{R3b.}] Let $\phi$ be finite. For each valuation $v$ 
an extension of $w$ to $E$, let $\phi_v: \resf{w} \to \resf{v}$
be the induced map. Then
\[
\res{v} \comp \phi^* \sum_{v} \comp \res{v}.
\]

\item[\textbf{R3c.}] Suppose $v$ restricts to a trivial valuation
$w$ on $F$. Then
\[
\res{v} \comp \phi_* = 0
\]

\item[\textbf{R3d.}] Suppose $v$ again restricts to a trivial 
valuation, and let $\phib: F \to \resf{v}$ be the induced map, and
$\idprime{p}$ a prime of $v$. Then
\[
\Special{v}{\idprime{p}} \comp \phi_* = \phib_*
\]

\item[\textbf{R3e.}] Let $u$ be a $v$-unit, and $\rho \in M(F)$,
one has
\[
\res{v}(\{u\} \cdot \rho) = -\{\overline{u}\} \cdot \res{v}(\rho).
\]
\end{enumerate}
\end{definition}

\begin{definition}
A homomorphism $\omega: M \to M'$ of cycle premodules over 
$\fields(B)$ of even (resp. odd) type is given by homomorphisms:
\[
\omega_F: M(F) \to M'(F)
\]
such that for each $\phi: F \to E$
\begin{enumerate}
\item $\phi_* \comp \omega_F = \omega_E \comp \phi_*$

\item $\phi^* \comp \omega_E = \omega_F \comp \phi^*$

\item for $\{a\} \in \milK_*F$ and $\rho \in M(F)$, $\{a\} \cdot 
\omega_F(\rho) = \omega_F(\{a\} \cdot \rho)$ (resp. 
$\{a\} \cdot \omega_F(\rho) = -\omega_F(\{a\} \cdot \rho)$)

\item $\res{v} \comp \omega_F = \omega_{\resf{v}} \comp \res{v}$
(resp. $\res{v} \comp \omega_F = -\omega_{\resf{v}} \comp 
\res{v}$).
\end{enumerate}
\end{definition}

\begin{ex}
$\milK_*$ is a cycle premodule.
\end{ex}

For $X$ a $k$-scheme, let $\subsch{1}{X}$ denote the collection of 
codimension 1 subscheme. Write $\xi_X$ be the generic point of an
irreducible $X$ with $K_X = \O_X,\xi_X$. If $X$ is normal, then 
for $x \in \codim{1}{X}$, the local ring $\O_X,x$ is a valuation 
ring of $K_X$ with residue field $\resf{x}$. Write $M(x)$ for 
$M(\resf{x})$, and $\res{x}: M(\xi_X) \to M(x)$ for the 
restriction map.

Furthermore, for $x, y \in X$, let $Z$ be the closed subscheme 
determined by $x$, and $\overline{Z}$ be the normalization $Z$.
Define
\[
\ptres{x}{y}: M(x) \to M(y)
\]
by
\[
\ptres{x}{y} = 
\begin{cases}
0 & y \notin \subsch{1}{Z} \\
\sum_{z|y} \phi_{\resf{z},\resf{x}}^* \comp \res{z} & \textrm{otherwise}
\end{cases}
\]
In case $\ptres{x}{y}$ is nonzero, the sum is taken over all $z$
lying over $y \in \subsch{1}{Z}$, and
$\phi_{\resf{z},\resf{y}}^*$ is the corestriction map associated to 
the finite field extension $\resf{y} \to \resf{z}$.

\begin{definition}
A cycle module $M$ on $\fields(B)$ is a cycle premodule which
satisfies the following conditions:

\begin{enumerate}
\item[\textbf{(FD)}] \itemhead{Finite support of divisors.} 
$X$ be a normal scheme and $\rho \in M(\xi_X)$. Then $\res{x}: 
M(\xi_X) \to M(X)$ is 0 for all but finitely many $x \in 
\subsch{1}{X}$.

\item[\textbf{(C)}] \itemhead{Closedness.} If $X$ is an integral
local local of dimension 2 with closed point $x_0$, then the map 
from $M(\xi_X)$ to $M(x_0)$ given by
\[
\sum_{x \in \subsch{1}{X}} \ptres{x_0}{x} \comp \ptres{x}{\xi}
\]
is 0.
\end{enumerate}
\end{definition}

Let $\CycMod$ denote the category of cycle modules. In ---,
D\'eglise demonstrated that $\CycMod$ is categorically 
equivalent to the category of homotopic modules (see -- ), which
we represent by $\HI_*$. 

\vskip 10pt
The complex \[\gersseq{M}\]

\section{Cycle Module Functorialities}\label{sect_cm_funct}

We write $\gcmplx{M}$ for the Gersten complex associated to the 
cycle module $M$. Rost showed that associated to every flat 
morphism $Y \to X$, there exists a contravariant morphism 
$\gcmplx{M}(X) \to \gcmplx{M}(Y)$, and to every equidimensional 
proper morphism $Y \to X$, a covariant $\gcmplx{M}(Y) \to 
\gcmplx{M}(X)$.

In [Deg06, 3.18], Deglise extended Rost's original work and 
associated to all local complete intersections $f: Y \to X$ for
which $Y$ emits resolution of singularities(see Deg06 3.13) a 
\emph{Gysin Map} 
\[
   f^*: \gcmplx{M}(X) \to \gcmplx{M}(Y)
\]
which is a map in the derived category of abelian groups, given 
by the composition of a morphism with the formal inverse of a 
quasi-isomorphism. More precisely, the latter is a formal inverse 
of the morphism $p^*$ for $p$ the projection map of a vector 
bundle. 

The Gysin map satisfies the following properties:
\begin{enumerate}
\item Since $f$ is also flat, $f^*$ coincide with the flat 
pullback map described earlier.

\item If $g: Z \to Y$ is a local complete intersection with $Z$
emitting resolution of singularities, then $(fg)^* = g^*f^*$.
\end{enumerate}

In the case where $f$ is a regular closed immersion, the 
hypothesis that $Y$ emits resolution of singularities is 
unnecessary. The map $f^*$ can be defined by using deformation
of the normal cone, following the original idea of Rost (cf 
[Deg.06, 3.3]).

We will use the following result due to Rost ([Ros96, 12.4]),
which partially describes the Gysin morphism:

\begin{prop}\label{prop_2_3}
Let $X$ be an integral scheme with function field $E$, and let $i: 
Z \to X$ be the closed immersion of a regular irreducible 
principal divisor, parametrized by $\idprime{p} \in \O_X(X)$. Let 
$v$ be the valuation of $E$ corresponding to the divisor $Z$. 
Therefore, the morphism $i^*: \cmcohom^0(X; M) \to \cmcohom^0(Z; 
M)$ is the restriction of $\Special{v}{\idprime{p}}: M(E) \to 
M(\resf{v})$.
\end{prop}

To each cartesian square
\[
\begin{diagram}
Y'      & \rTo{j} & X'      \\
\dTo{g} &         & \dTo{f} \\
Y       & \rTo{i} & X
\end{diagram}
\]
such that $i$ is a regular closed immersion, we associate a 
\emph{refined Gysin morphism} $\refgysin$ in the derived 
category:
\[
\refgysin: \gcmplx{M}(X') \to \gcmplx{M}(Y')
\]
that satisfies the following properties:
\begin{enumerate}
\item If $j$ is regular and the morphism of the normal cones
\[
N_{Y'}(X') \to g^{-1}N_Y(X)
\] 
is an isomorphism, then $\refgysin = j^*$.

\item If $f$ is proper, then $i^*f_* = g_*\refgysin$.
\end{enumerate}

Further, if the canonical immersion of the cone $C_{Y'}(X') \to
g^{-1}N_Y(X)$ in the normal fiber of $i$ is of pure codimension
equal to $e$, then the morphism $\refgysin$ is of cohomological
degree $e$.

For all smooth couples $(X, Y)$ and for all finite correspondence
$\alpha \in \Cor(X, Y)$, we define a map in the derived category:
\[
\alpha^* : \gcmplx{M}(Y) \to \gcmplx{M}(X)
\]
(see Deg06, 6.9). We describe this map as follows. Suppose that 
$\alpha$ is the class of a closed irreducible subscheme $Z$ of $X 
\times Y$. Consider the morphisms:
\[
X \stackrel{p}{\rightarrow} Z \stackrel{i}{\to} Z \times X \times Y
\stackrel{q}{\to} Y
\]
where $p$ and $q$ denote the canonical projections and $i$, the
graph of the closed immesion $Z \to X \times Y$. Therefore,
\begin{equation}\label{eq_2_5_a}
\alpha = p_* i^*q^*
\end{equation}
where $i^*$ denotes the Gysin map of a regular closed immersion 
$i$, $q^*$ the flat pullback and $p_*$ the finite push out.

The property $(\beta\alpha)^* = \alpha^*\beta^*$ is proved in 
[Deg06, 6.5].

\subsection{Localization Exact sequence}

We recall the localization exact sequence, following [Ros96],
and prove a supplementary result concerning its functoriality.
For a closed immersion $i: Z \to X$ purely of codimension $c$,
let $j: U \to X$ be the open immersion associated to its 
complement. By using functoriality mentioned above, we obtain
a short exact sequence of complexes:
\begin{equation}\label{eq_loc_exact_seq}
0 \to C_M^{p - c}(Z)_{n - c} \stackrel{i_*}{\to}
      C_M^p(X)_n \stackrel{j_*}{\to}
      C_M^p(U)_n \to 0.
\end{equation}
from which we have the following long exact sequence:
\begin{equation}\label{eq_loc_long_exact_seq}
\cdots \to \cmcohom^{p - c}_M(Z)_{n - c} \stackrel{i^*}{\to}
           \cmcohom^p_M(X)_n \stackrel{j^*}{\to}
           \cmcohom^p_M(U)_n \stackrel{\ptres{Z}{U}}{\to}
           \cmcohom^{p - c + 1}_M(Z)_{n - c} \to \cdots
\end{equation}
where the morphism $\ptres{Z}{U}$ is defined for the associated
graded by
\[
   \sum_{x \in \subsch{p}{U}, z \in \subsch{p - c + 1}{Z}} 
      \ptres{z}{x}.
\]
This sequence is natural with respect to proper and flat pushout.
We have the following proposition:

\begin{prop}\label{prop_2_6}
Consider the following cartesian square:
\[
\begin{diagram}
T       & \rTo{\iota'} & Z       \\
\dTo{k} &              & \dTo{i} \\
Y       & \rTo{\iota}  & X
\end{diagram}
\]
for which $\iota$ is a regular closed immersion. Suppose that $i$
(resp $k$) is a a closed immersion complementary an open 
immersion $j: U \to X$ (resp $l: V \to X$). Let $h: V \to U$ be
the map induced by $\iota$. Finally, suppose that $i$ (resp. $k$)
is of pure codimension equal to $c$ (resp. $d$). Then, the 
following diagram is commutative
\[
\begin{diagram}
\cdots & \rTo & \cmcohom^{p - c}_M(Z)_{n - c} & \rTo{i_*} & 
   \cmcohom^{p}_M(X)_n & \rTo{j^*} & \cmcohom^p_M(U) & 
   \rTo{\ptres{Z}{U}} & \cmcohom^{p - c + 1}_M(Z)_{n - c} & \rTo 
   & \cdots \\
       &      & \dTo{\refgysin}               &           & 
       \dTo{\iota^*}        &           & \dTo{h^*}       &                    
       & \dTo{\refgysin}                   \\
\cdots & \rTo & \cmcohom^{p - d}_M(T)_{n - d} & \rTo{i_*} & 
   \cmcohom^{p}_M(Y)_n & \rTo{j^*} & \cmcohom^p_M(V) & 
   \rTo{\ptres{T}{V}} & \cmcohom^{p - d + 1}_M(T)_{n - d} & \rTo 
   & \cdots
\end{diagram}
\]
\end{prop} 

\begin{rmk}
We could generalize the previous proposition to the case of 
refined Gysin maps as in Prop. 4.5 of [Deg06]. We leave this as a 
task to the reader.
\end{rmk}

\begin{rmk}
While the hypothesis of $i$ having pure codimension is natural, 
the one about $k$ is not, especially in the case where $Y$ does 
not intersect $Z$ transversely. Here, the hypothesis serves only
to determine the cohomological degree of the maps, and the 
hypothesis could easily be replaced by the assumption that the
maps are homogeneous with respect to cohomology degree.
\end{rmk}

\begin{proof}
It suffice to resume the proof of Proposition 4.5 of [Deg06]
for the following case:
\[
\begin{diagram}
T               & \rTo        &   & Z \\
\dTo(0,4)       & \rdTo{k}    &   & \dLine(0,4) &\rdTo{i}   \\
                &             & Y & \rTo        &           & X \\
                & \ldEquals   &   & \dTo        & \ldEquals \\
Y               & \rTo{\iota} &   & X
\end{diagram}
\]
In this case, we have the following commutative diagram, with
analogous notation as \emph{loc. cit.}
%\[
%\begin{diagram}
%\end{diagram}
%\]
The squares (1), (2), and (3) are commutative by similar reasons
as given in \emph{loc. cit.}, and (1'), (2') and (3') are 
clearly commutative. Each map that appear in the diagram are
well-defined maps of complexes, and induce maps between long exact
localization sequences.

To conclude, it suffices to point out that the maps $p^*$,
$p_T^*$ and $p_U^*$ are quasi-isomorphisms.
\end{proof}

\begin{cor}\label{cor_2_8}
Consider the following cartesian square:
\[
\begin{diagram}
T       & \rTo{g}      & Z       \\
\dTo{k} &              & \dTo{i} \\
Y       & \rTo{f}      & X
\end{diagram}
\]
of smooth schemes such that $i$ (resp. $k$) is a closed immersion 
of pure codimension equal to $c$, with a complement open immersion 
$j: U \to X$ (resp. $l: V \to X$). Let $h: V \to U$ represent the 
map induced by $f$. Then, the following diagram is commutative:
\[
\begin{diagram}
\cdots & \rTo & \cmcohom^{p - c}_M(Z)_{n - c} & \rTo{i_*} & 
   \cmcohom^{p}_M(X)_n & \rTo{j^*} & \cmcohom^p_M(U) & 
   \rTo{\ptres{Z}{U}} & \cmcohom^{p - c + 1}_M(Z)_{n - c} & \rTo 
   & \cdots \\
       &      & \dTo{g^*}                     &           
       & \dTo{f^*}           &           & \dTo{h^*}       
       &                    & \dTo{g^*}                         \\
\cdots & \rTo & \cmcohom^{p - c}_M(T)_{n - c} & \rTo{i_*} & 
   \cmcohom^{p}_M(Y)_n & \rTo{j^*} & \cmcohom^p_M(V) & 
   \rTo{\ptres{T}{V}} & \cmcohom^{p - c + 1}_M(T)_{n - c} & \rTo 
   & \cdots
\end{diagram}
\]
\end{cor}

\begin{rmk}\label{rmk_2_9}
In [Deg08B], a \emph{closed pair} is a couple $(X, Z)$ such that 
$X$ is a smooth scheme and $Z$ is a closed subscheme. We say that 
$(X, Z)$ is \emph{smooth} (resp. \emph{of codimension $n$}) if 
$Z$ is smooth (resp.  purely of codimension $n$ in $X$).  

If $i: Z \to X$ is the associated closed immersion, a 
\emph{morphism of closed pairs} $(f, g)$ is a commutative square
\[
\begin{diagram}
T       & \rTo{g}      & Z       \\
\dTo{k} &              & \dTo{i} \\
Y       & \rTo{f}      & X
\end{diagram}
\]
which is topologically cartesian. We say that $(f, g)$ is 
\emph{cartesian} (resp. \emph{transverse}) when the square is
cartesian (resp. and the morphism induced on the normal cones
\[
C_T Y \to g^{-1}C_Z X
\]
is an isomorphism.)

The preceding corollary shows that the localization sequence 
associated to a cycle module $M$ and a closed pair $(X, Z)$ is 
natural with respect to transverse morphisms.
\end{rmk}

\subsection{Associated homotopy module}\label{subsect_assoc_hm}

Let $M$ be a cycle module, and write 
\[
M^p(X) = \bigoplus_{x \in \subsch{p}{X}} M(\resf{x}).
\] 
By ---, for all scheme $X$ essentially of finite type, the 
morphism
\[
d_M^p = \sum_{x \in \subsch{p}{X}, y \in \subsch{p + 1}{X}} 
\ptres{x}{y} : M^p(X) \to M^{p + 1}(X)
\]
is well defined, and fits into a chain complex
\[
\cdots \to M^p(X) \stackrel{d_M}{\to} M^{p + 1} \to \cdots
\]
Write $\cmcohom^p(X; M)$ for the $p$th cohomology group of the 
complex.

According to [Deg2.5], $\cmcohom{0}(-,M)$ defines a graded 
presheaf with transfers. According to [Deg06, 6.9], presheaf is 
in fact a sheaf. We write $F^M$ for the corresponding sheaf, and 
define a homotopic module strucutre as follows.

Let $X$ be a smooth scheme. Begin by considering the long exact
localization sequence (Deg.2.5c) associated to the zero section of
$X \to X \times \A^1$:
\[
0 \to F^M_n(X \times \A^1) \stackrel{j}{\to} F^M_n(\Gmpt \otimes X) 
   \stackrel{d}{\to} F^M_{n - 1}(X) \to \cdots
\]
where the map $j$ is given by the inclusion of $\Gmpt$ into $\A^1$,
and $d$ is the map induced by the degree $-1$ map
\[
\ptres{E}{0}: M(E(t)) \to M(E)
\]
associated to every valuation of $E(t)$ for every generic point 
$E$ of $X$.

Let $s_1: X \to \Gmpt \times X$ be the unit section. Recall that
$(F^M_n)_{-1}(X) = \ker (s_1^*)$. In fact, since $F^M_n$ is 
homotopy invariant, the canonical morphism $\ker s_1^* \to
\cok j^*$ is an isomorphism. Therefore, the map $\ptres{0}{X}$ 
induces a map
\[
\loopCM_n : (RF^M_{n})(X) \to F^M_{n - 1}(X).
\]
First, the localization sequence, introduced earlier, is 
compatible with the transfers on $X$. This follows from 2.5 and 
corollary 2.8. Therefore, $\loopCM_n$ defines is a map between 
homotopy invariant sheaves with transfers. Further, for all 
function field $E$, $\cmcohom(\A^1_E; M) = 0$ [Ros96, (2.2)H]. 
Thus, the fiber of $\loopCM_n$ in $E$ is an isomorphism, and by 
1.4, an isomorphism between homotopy invariant sheaves.

Therefore, $(F^M_*, \loopCM^{-1})$ defines a homotopic module that
which depends functorially on $M$.

\section{Equivalence of Categories between Homotopic Modules and 
Cycle Modules}

\subsection{Generic transformations} Consider a couple $(E, n)$
of a function field $E$ and an integer $n$. Recall that in 
[Deg08B 3.3.1] we have associated to $(E, n)$ a \emph{generic 
motive}
\[
M(E)\{n\} = ``\varprojlim_{A \subset E}'' M(\spec(A))\{n\}
\]
in the category of pro-objcts of $\DM_{gm}$. We write 
$\GenMot$ for the category of generic motives.

Let $(F_*, \loopCM_n)$ be a homotopic module. Recall from -- that
$(F_*, \loopCM_n)$ can be identified with a functor $\phi: \DM_{gm}
\to \Ab$, and let $\widehat{\phi}$ be its canonical extension to
the category of pro-$\DM_{gm}$. As established in [Deg08B, 6.2.1],
the restriction of $\widehat{\phi}$ to $\GenMot$ is a cycle module
that we write as $\assocCM{F_*}$, and that we call the generic
transformation of $F_*$.

We reiterate some key parts of its construction from [Deg08B]. 
First, note that for all generic motif $M(E)\{n\}$, 
$\widehat{\phi}(M(E)\{n\}) = \assocCM{F_{-n}}(E)$ is nothing
but the fiber of $F_{-n}$ at $E$ (cf. 1.4). We can also interpret
the generic transformation $\assocCM{F}$ as a system of fibers 
of $F$, together with specialization morphisms between them,
which satisfy the following structure axioms of cycle premodules:

\begin{enumerate}
\item[\textbf{D1.}] Functoriality (which is clear).

\item[\textbf{D2.}] Following [Deg08B 5.2], for a finite extension
$E \to L$, let $X$ and $Y$ be models for $E$ and $L$ respectively.
It follows that there exists a finite surjective morphism $f: Y 
\to X$ for which the induced map on the function fields is 
precisely that of $E \to L$.

The graph of $f$ regarded as a cycle of $X \times Y$ define a 
finite correspondence from $X$ to $Y$. This in turn yields a map 
$F_*(X) \to F_*(Y)$. We can show that this map is compatible with 
the restriction to an open subscheme of $X$, and induces the 
desired functoriality.

\item[\textbf{D3.}] Following [Deg08B 5.3], let $E$ be a function
field and $x \in E^{\times}$ is any unit. Let $X$ be a model of 
$E$, equipped with an invertible section $X \to \Gmpt$ corresponding
to $x$. Consider the closed immersion $s_x: X \to \Gmpt \times X$
induced by this section. From this setup, we obtain the following 
morphism:
\[
\gamma_x: F_{n - 1}(X) \stackrel{\loopCM_{n - 1}}{\to} (F_n)_{-1}(X)
\stackrel{\nu}{\to} F_n(\Gmpt \times X) \stackrel{s_x^*}{\to} 
F_n(X)
\]
where $\nu$ is the canonical inclusion. This morphism is 
compatible with the restriction to an open subscheme of $X$, and 
induced the desired morphism for $\assocCM{F_*}$.

\item[\textbf{D4.}] Following Deg08B, 5.4, Let $E$ be a function 
field, and $v$ a valuation on $E$. We could find a smooth scheme
$X$ together with a codimension 1 point $x$ such that the induced
reduced structure $Z$ associated to $x$ in $X$ is smooth with 
local ring $\O_{X,x}$ isomorphic to the ring of integers of $v$.

Let $U = X - Z$ with open immersion $j: U \to X$. Recall that the
motive $M_Z(X)$ of the pair $(X, Z)$ is defined as an object of
$\DMeff_{gm}$ represented by the complex concentrated in degree 0
and $-1$ with a single nonzero differential given by $j$. This
motive fits into the following distringuished triangle:
\[
M_Z(X)[-1] \stackrel{\res{X, Z}'}{\to} M(U) \stackrel{j*}{\to}
M(X) \stackrel{+1}{\to} M_Z(X).
\]
We have defined in Deg08B, 2.2.5 a purity isomorphism as follows:
\[
\purity{X, Z}: M_Z(X) \to M(Z)(1)[2].
\]
From this, we obtain the following map (which we write as 
$\res{X, Z}$) given by:
\begin{align*}
F_n(U) &= \phi_n(M(U)) \xrightarrow{\phi_n(\res{X,Z}')} 
   \phi_n(M_Z(X)[-1]) \\
   &\xrightarrow{\phi_n(\purity{X, Z}^{-1})} \phi_n(M(Z)(1)[2])
    = (F_n)_{-1}(Z) \stackrel{\loopCM_n^{-1}}{\to} F_{n - 1}(Z),
\end{align*}
where we write $\phi_n(\mathcal{M}) = \phi(\mathcal{M}(n)[2n])$
for a motive $\mathcal{M}$. The residue morphism of the cycle
module $\assocCM{F_*}$ is given by the inductive limit of the
morphisms $\res{U, Z \cap U}$ given by the neighborhood $U$ of
$x \in X$.
\end{enumerate}

\section{Gersten Resolution: Functoriality I}

Consider a cycle module $M$ and its associated homotopy module 
$F^M$ (see --). Here, and in the following, we will drop the 
gradation on homotopy modules to simplify notation. By 
[Ros96, 6.5], we have a canonical isomorphism for all smooth 
schemes $X$ and integer $p \in \Z$:
\[
\cmcohom^p(X; M) = \hzar^p(X; F^M).
\]
We recall the construction of this isomorphism all in generalizing
this to the case of the Nisnevich topology. Let $X$ be a smooth 
scheme and $X_{\Nis}$ be the small Nisnevich site associated to 
$X$. The morphisms of $X_{\Nis}$ being \'etale, we obtain the 
following presheaf of complexes of abelian groups on $X_{\Nis}$ 
by recalling the functoriality properties in Deg08A 2.1:
\[
M^*_X: V/X \mapsto \gcmplx{M}(V).
\]
We show that this is a Nisnevich sheaf [see Deg08B, proof of 6.10].
Furthermore, set
\[
F^M_X = \sheaf{H}^0(M^*_X).
\]
Therefore, $F^M_X$ is the restriction of the Nisnevich sheaf $F^M$,
defined on the large Nisnevich site $\SmSite{k}$ to the small site
$X_{\Nis}$. By [Ros96, 6.1], the map
\[
F^M_X \to M^*_X
\]
is a quasi-isomorphism. \footnote{The complex of sheaves $M^*$ is 
the Gersten resolution of the sheaf $F^M$. It's the Nisnevich 
version of the resolution of Cousin in [Har66].} Therefore, it 
induces an isomorphism
\[
\hnis^p(X; F^M_X) \to \hnis^p(X; M^*_X).
\]
As detailed in [Deg08B, 6.10], the complex $M^*$ satisfies the
properties of Brown-Gersten in the sense of [CD09A, 1.1.9]. From 
the proof of \emph{loc. cit} 1.1.10, we have that the canonical
map
\[
H^p(\gcmplx{M}(X)) \to \hnis^p(X; M^*_X)
\]
is an isomorphism. The two isomorphisms define:
\begin{equation}\label{eq_cm_hnis_equiv}
\rho_X: \cmcohom^p(X; M) = H^p(\gcmplx{M}(X)) \stackrel{\sim}{\to}
   \hnis^p(X; F^M_X) \simeq \hnis(X; F^M).
\end{equation}

\begin{lem}\label{lem_cm_hnis_equiv_morph}
The isomorphism $\rho_X$ defined in Equation 
\ref{eq_cm_hnis_equiv} is natural in $X$ with respect to any
scheme morphisms.
\end{lem}
\begin{proof}
Note that, since $F^M_X$ is a sheaf on $\SmSite{k}$, for all 
morphisms $f: Y \to X$ of smooth schemes, we have a canonical
natural transformation1:
\[
F^M_X \to f_*F^M_Y
\]
which induces the following map of the derived category:
\[
\tau_f: F^M_X \to Rf_*F^M_Y.
\]
The proof consists of lifting this transformation to the level
of the resolution $M^*_X$.

We first consider the case where $f$ is flat. Following 
[Ros96, (3.4)] we have the following map between the complexes:
\[
f^*: \gcmplx{M}(X) \to \gcmplx{M}(Y)
\]
which is natural in $X$ with respect to \'etale morphisms. The
corresponding natural transformation on $X_{\Nis}$ define a map
in the derived category of abelian sheaves on $X_{\Nis}$:
\begin{equation}\label{eq_3_3_a}
\eta_f: M^*_X \to f_*M^*_Y = Rf_*M^*_Y.
\end{equation}
The last equality results from the fact tat $M^*(Y)$ satisfies
the property of Brown-Gersten. By definition of the structure of
the sheaf $F^M$, the following diagram commutes:
\[
\begin{diagram}
F^M_X  & \rTo{\tau_f} & Rf_* F^M_Y \\
\dTo   &              & \dTo       \\
M^*_X & \rTo{\eta_f} & Rf_*M^*_Y.
\end{diagram}
\]
In this case, it follows the $\rho$ defined in the statement of
the lemma is natural with respect to flat morphisms. We remark
that if $f$ is the projection of a vector bundle, then $\eta_f$
is a quasi-isomorphism.

It remains to consider the case where $f$ is a closed immersion
$Z \to X$ of smooth schemes. Let $N$ be the normal bundle 
assocaited to $i$. The specialization of the normal bundle defined
by Rost (cf. [Ros96, sec 11]) is a morphism of complexes
\[
\sigma_ZX: \gcmplx{M}(X) \to \gcmplx{M}(N)
\]
which is natural in $X$, in particular, with respect to \'etale
morphisms (cf. [Deg06, 2.2]). We write $\nu$ for the composition
\[
N \stackrel{p}{\to} Z \stackrel{i}{\to} X.
\]
In this case, we obtain the following canonical map in the derived 
category
\[
\sigma_i: M^*_X \to R\nu_*M^*_N.
\]
Since the map $\eta_p$ is a quasi-isomorphism, we have therefore a
canonical map in the derived category
\begin{equation}\label{eq_3_3_b}
\eta_i: M^*_X \to Ri_*M^*_Z.
\end{equation}
Lastly, recall that, by definition of the pull-back on $F^M$, the
following diagram is commutative:
\[
\begin{diagram}
F^M(X)        & \rTo{i^*}         &               &           & F^M(Z)        \\
\dTo          &                   &               &           & \dTo          \\
\gcmplx{M}(X) & \rTo{\sigma_Z(X)} & \gcmplx{M}(N) & \lTo{p^*} & \gcmplx{M}(Z).
\end{diagram}
\]
from which, we see that the following diagram is commutative:
\[
\begin{diagram}
F^M_X & \rTo{\tau_i} & Ri_*F^M_Z \\
\dTo  &              & \dTo      \\
M^*_X & \rTo{\eta_i} & Ri_*M^*_Z
\end{diagram}
\]
which concludes the proof.
\end{proof}

\begin{rmk}
We will generalize this lemma to the case of finite correspondences
in proposition 3.10.
\end{rmk}

\begin{rmk}
The constructions given in Equations (\ref{eq_3_3_a}) and 
(\ref{eq_3_3_b}) of the preceding proof give us a way to associate
to all scheme maps $f: Y \to X$ a commutative diagram in the 
derived category of sheaves on $X_{\Nis}$:
\[
\begin{diagram}
F^M_X & \rTo{\tau_f} & Rf_*F^M_Y \\
\dTo  &              & \dTo      \\
M^*_X & \rTo{\eta_f} & Rf_*M^*_Y,
\end{diagram}
\]
obtained by considering the factorization of $f$ by the regular
immersion of $Y$ into the graph of $f$ followed by a projection. 
We can show that $\eta_f$ is compatible with the composition of 
these maps.
\end{rmk}

We resume the notations used earlier in 3.2. Consider the 
realization functor
\[
\phi : \DM_{gm}^{op} \to \Ab
\]
assocaited to the homotopy module $F^M$ defined in section 1.3. 
By definition, the isomorphism $\rho_X$ corresponds to an 
isomorphism
\[
\cmcohom^p(X; M)_n \to \phi_n(M(X)[-p]).
\]
Furthermore, consider a closed immersion $i: Z \to X$ of smooth 
schemes and $j: U \to X$ the open immersion complement to $i$. 
Suppose that $i$ is of pure codimension equal to $c$. We see that
the localization exact sequence in Equation 
(\ref{eq_loc_exact_seq}) gives rise to the following map, 
represented by the dotted arrow, that makes the following diagram
commutative:
\[
\begin{diagram}
0 & \rTo & \gcmplx{M}(Z)_{n - c}[-c] & \rTo{i^*}            
  & \gcmplx{M}(X)_n     & \rTo{\;\; j^*\;\;} & \gcmplx{M}(U)_n     
  & \rTo & 0 \\
  &      & \dDotsto{(1)}             &                      
  & \dTo                &           & \dTo                \\
0 & \rTo & R\Gamma_Z(X, M^*_X)_n     & \rTo{\phantom{aaaa}} 
  & R\Gamma(X, M^*_X)_n & \rTo{\;\; j^*\;\;} & R\Gamma(U, M^*_X)_n 
  & \rTo & 0.
\end{diagram}
\]
Here, we again use the fact that the complex $M^*_X$ satisfy the
properties of Brown-Gersten.

The arrow (1) is a quasi-isomorphism by the five lemma, since so 
are the other two vertical maps. Consider the motive $M_Z(X)$ 
associated to the closed pair $(X, Z)$ -- cf. 3.1 D4. Via the 
quasi-isomorphism (1) and the canonical identification of
$H_Z^p(X; F^M)_n$ with $\phi_n(M_Z(X)[-p])$, we obtain the 
following commutative diagram:
\[
\begin{diagram}
\cmcohom^{p - 1}(U; M)_n & \rTo{\;\;\;\ptres{Z}{U}\;\;\;} 
                         & \cmcohom^{p - c}(Z; M)_{n - c} 
                         & \rTo{\;\;\;i^*\;\;\;} 
                         & \cmcohom(X; M)_n \\
\dTo{\rho_U}             &                                
                         & \dTo{\rho'_{X, Z}}                
                         &                       
                         & \dTo{\rho_X}        \\
\phi_n(M(U)[-p]          & \rTo                           
                         & \phi_n(M_Z(X)[-p])             
                         & \rTo                  
                         & \phi_n(M(X)[-p])
\end{diagram}
\]
in which the vertical arrows are isomorphisms. The map 
$\rho'_{X,Z}$ is natural in $(X, Z)$ with respect to transverse 
maps (defined in Remark \ref{rmk_2_9}). This in effect follows
from Corollary \ref{cor_2_8}, or more precisely from the 
commutative diagram appearing in the proof of Proposition, 
\ref{prop_2_6} using on the one hand the uniqueness of the map (1), 
and on the other hand the description of the functoriality derived 
from $M^*_X$ established here --- i.e. the natural transformations 
$\tau_f$ and $\tau_i$.

As consequence of this construction, we have the following key 
lemma:

\begin{lem}\label{lem_3_6}
Consider the Gysin triangle (cf. [Voev00B, 3.5.4]) associated to
a closed pair $(X, Z)$ with $i: Z \to X$ a closed immersion of
pure codimension equal to $c$ with complement open immersion $U 
\to X$:
\[
M(U) \to M(X) \stackrel{i^*}{\to} M(Z)(c)[2c] 
   \stackrel{\res{X, Z}}{\to} M(U)[1].
\]
Then the following diagram is commutative:
\[
\begin{diagram}
\cmcohom^{p - 1}(U; M)_n & \rTo{\ptres{Z}{U}}      
   & \cmcohom^{p - c}(Z; M)_{n - c} & \rTo{i^*}         
   & \cmcohom^{p}(X; M) \\
\dTo{\rho_U}             &                         
   & \dTo{\rho_Z}                   &                   
   & \dTo{\rho_X}       \\
                         &                         
                         & \phi_{n - c}(M(Z)[c - p])      \\
                         &                         
                         & \dEquals                       \\
\phi_n(M(U)[-p])         & \rTo{\phi_n{\res{X,Z}}} 
   & \phi_n(M(Z)(c)[2c - p])        & \rTo{\phi_n(i^*)} 
   & \phi_n(M(X)[-p]).
\end{diagram}
\]
\end{lem}
\begin{proof}
We use the construction of the Gysin triangle defined in [Deg08B].
Consider the purity isomorphism defined in [Deg08B, 2.2.5]
\[
\purity{X, Z}: M_Z(X) \to M(Z)(c)[2c].
\]
By the preceding paragraphs, the isomorphism
\begin{align*}
\rho_{X, Z}: \cmcohom^{p - c}(Z; M)_{n - c} &\xrightarrow{\rho'_{X, Z}} \phi_n(M_Z(X)[-p]) \\
&\xrightarrow{\phi(\purity{X, Z}} \phi_n(M(Z)(c)[2c - p]) = \phi_{n - c}(M(Z)[c - p])
\end{align*}
fits into the commutative diagram:
\[
\begin{diagram}
\cmcohom^{p - 1}(U; M)_n & \rTo{\ptres{Z}{U}}      
   & \cmcohom^{p - c}(Z; M)_{n - c} & \rTo{i^*}         
   & \cmcohom^{p}(X; M) \\
\dTo{\rho_U}             &                         
   & \dTo{\rho_{X, Z}}              &                   
   & \dTo{\rho_X}       \\
                         &                         
                         & \phi_{n - c}(M(Z)[c - p])      \\
                         &                         
                         & \dEquals                       \\
\phi_n(M(U)[-p])         & \rTo{\phi_n{\res{X,Z}}} 
   & \phi_n(M(Z)(c)[2c - p])        & \rTo{\phi_n(i^*)} 
   & \phi_n(M(X)[-p]).
\end{diagram}
\]
It suffices to show that $\rho_{X, Z} = \rho_Z$. Once again, by 
the comments preceding the lemma, the map $\rho_{X, Z} - \rho_Z$
is natural in $(X, Z)$ with respect to transverse maps. Let 
$P_Z X$ be the projective closure of the normal bundle of $Z$ in 
$X$. Consider the blow-up $B_Z(\A^1 \times X)$ of $Z \times \{0\}$
in $X$, together with associated deformation diagram
\[
(X, Z) \xrightarrow{(d, i_1)} (B_Z(\A^1 \times X), \A^1 \times Z) 
   \xleftarrow{(d', i_0)} (P_Z X, Z).
\]
The commutative squares corresponding to $(d, i_1)$ and 
$(d', i_0)$ are transverse. Therefore, it suffices to consider 
only the case where $(X, Z) = (P_Z X, Z)$. In this setting, the
closed immersion $i$ admit a retraction and the map $\rho_{X, Z}$
(resp $\rho_Z$) is determined uniquely by $\rho_X$.
\end{proof}

We are now ready to prove the main theorem on the correspondence 
between cycle modules and homotopy modules.

\begin{thm}\label{thm_corresp_cyc_mod_hom_mod}
The functors
\begin{align*}
\HI_* &\leftrightarrows \CycMod \\
  F_* & \mapsto \assocCM{F_*}   \\
F^M_* & \mapsfrom M
\end{align*}
are equivalence of categories from one to the other.
\end{thm}
\begin{proof}
It suffices to construct the two natural isomorphisms which
realize equivalence.

\textbf{First isomorphism}: Fix a cycle module $M$, and $F^M_*$ the
associated homotopy module. By definition, for all function field
$E$, there exists canonical maps
\[
a_E: \assocCM{F^M_n}(E) = \varinjlim_{A \subset E} 
   \cmcohom^0(\spec(A); M)_n \to M_n(E).
\]
That this is an isomorphism is clear, and it remains to show that
$a$ defines a morphism of cycle modules. The compatibility with 
(D1) is evident. The compatibility with D2 follows from the fact 
that for all finite surjective maps $f: Y \to X$, the map 
$\cmcohom^0(f; M)$ is the proper pushout $f_*$ (cf. Deg08B, 6.6).

\emph{Compatibility to (D3) : } resuming the setup of (D3) of
(3.1) for the homotopy module $F^M_*$ and for a unit $x \in 
E^\times$. We consider the canonical map
\[
a_E': \assocCM{F^M_n}(\Gmpt \times E) = \varinjlim_{A \subset E}
\cmcohom^0(\spec(A[t, t^{-1}]); M)_n \to M_n(E(t)).
\]
For all $E$-point $y$ of $\spec E[t]$, let $v_y$ denote the 
corresponding valuation of $E(t)$ with uniformizing parameter
$t - y$. By Proposition \ref{prop_2_3}, the following diagram is
commutative:
\[
\begin{diagram}
\assocCM{F^M_n}(\Gmpt \times (E))   & \rTo{\Special{x}{*}}        & \assocCM{F^M_n}(E)   \\
\dTo{a'_E}                          &                             & \dTo{a_E}            \\
\phi_n(E(t))                        & \rTo{\Special{v_x}{t - x}} & \phi_n(E).
\end{diagram}
\]
By the definition of structure map $\loopCM_*$ of $F^M_*$ (cf. 
Subsection \ref{subsect_assoc_hm}), the map $\nu': 
\assocCM{F^M_{n - 1}}(E) \xrightarrow{\epsilon_n - 1} 
(\assocCM{F^M_n})_{-1}(E) \stackrel{\nu}{\to} 
\assocCM{F^M_n}(\Gmpt \times (E))$ is the section of the short 
exact sequence
\[
0 \to \assocCM{F^M_n}(E) \stackrel{p^*}{\to} 
   \assocCM{F^M_n}(\Gmpt \times (E)) \stackrel{\partial}{\to}
   \assocCM{F^M_{n - 1}}(E) \to 0
\]
which correspond to the retraction $s_1^*$ of $p^*$, for $s_1: 
(E) \to \Gmpt \times (E)$ the unit section of the projection $p: 
\Gmpt \times (E) \to (E)$. In particular, $\nu'$ is characterized 
by the prperties $\partial\nu' = 1$ and $s_1^*\nu' = 0$.

Let $\iota: E \to E(t)$ denote the canonical inclusion. Using the
cycle premodule relations, we can verify the following facts:
\begin{enumerate}
\item[(1)] for all $\rho \in M_n(E)$, 
   $\res{v_0}(\{t\} \cdot \iota_*(\rho)) = \rho$.

\item[(2)] for all $y \in E^\times$ and $\rho \in M_n(E)$, 
   $\res{v_y}(\{t\} \cdot \iota_*(\rho)) = 0$.

\item[(3)] for all $y \in E^\times$ and $\rho \in M_n(E)$,
   $\Special{v_y}{t - y}(\{t - y\} \cdot \iota_*(\rho) = \{y\} 
      \cdot \rho$.
\end{enumerate}
By (2), the map $M_n(E) \to M_n(E(t))$, given by $\rho \mapsto 
\{t\} \cdot \iota_*(\rho)$ induces a unique map in the following
commutative diagram as the dotted arrow in the first line:
\[
\begin{diagram}
\assocCM{F^M_n}(E) & \rDotsto                  & \assocCM{F^M_n}(\Gmpt \times (E)) \\
\dTo{a_E}          &                           & \dTo{a'_E}                        \\
M_n(E(t))          & \rTo{\{t\} \cdot \iota_*} & M_n(E).
\end{diagram}
\]
By the relations (1) and (3), with $y = 1$, the dotted arrow 
satisfy the two properties characterizing $\nu'$. Hence we 
conclude from (3) with $y = x$ that $\nu' \comp s_x^*(\rho) = 
\{x\} \cdot \rho$ which proves the desired relation for 
(D3).

\emph{Compatibility with (D4) : } Recall the setup of (D4) in 3.1:
$E$ a function field, $v$ a valuation $x$ a codimension 1 point 
of a smooth scheme $X$ corresponding to a smooth subvariety $Z$ 
for which $\O_{X, x}$ is isomorphic to the integer ring of $v$. 
The compatibility with the residue is there a direct consequence 
of Lemma \ref{lem_3_6} applied to all neighborhood $U$ of $x$ in 
$X$ of the closed immersion $i: Z \cap U \to U$ in the case 
$c = 1, p = 1$.

\textbf{Second isomorphism} : Consider a homotopy module $(F_*, 
\loopCM_*)$. For all smooth scheme $X$, by considering the 
inductive limit of restriction maps $F(X) \to F(U)$ for open 
subschemes $U$ of $X$, we obtain a map $F_*(X) \to 
\gcmplx{\assocCM{F_*}}(X)$. By the definition of differential,
this map induces a homogeneous map $b_X: F_*(X) \to \cmcohom^0(X; 
\assocCM{F_*})$ of degree 0.

The key point is to show that this map is natural with respect to
finite correspondences. Let $\alpha \in \Cor(X, Y)$ be a finite 
correspondence between smooth schemes. We can assume in addition 
that $X$ and $Y$ are connected. Recall that for all open dense 
$j: U \to X$, the map $j^* : \cmcohom^0(X; \assocCM{F_*}) \to 
\cmcohom^0(U; \assocCM{F_*})$ is injective by the localization 
exact sequence (\ref{eq_loc_long_exact_seq}). Therefore, we can 
replace $\alpha$ by $\alpha \comp j$ and $X$ by $U$.

By additivity, we can further reduce to the case where $\alpha$ is
the class of integral closed subschemes of $X \times Y$ finite 
surjective over $X$. In this case, $\alpha \comp j = [Z \times_X 
U].$ Since $k$ is perfect, by shrinking $X$ if necessary, we can
assume that $Z$ is smooth over $k$. Recall $\alpha^* = p_*i^*q^*$ 
(see Equation \ref{eq_2_5_a}) for the following setup:
\[
X \stackrel{p}{\longleftarrow} Z \stackrel{i}{\rightarrow} 
   Z \times X \times Y \stackrel{q}{\rightarrow} Y.
\]
We are therefore left to prove naturality in each of the following
three cases:

\emph{First case : } If $\alpha = q$ is a flat morphism, 
compatibility follows from the definition of flat pull-back of
$\cmcohom^0(\cdot; \assocCM{F_*})$ from the definition of (D1).

\emph{Second case : } If $\alpha = \Gamma_p$, where $p : Z \to X$
is a finite surjective map of smooth schemes, and $\Gamma_p$ is
the graph of $p$, viewed as a finite correspondence from $X$ to 
$Y$ (see D2). In this case, naturality follows from the definition 
of proper pushout on $\cmcohom^0$ and from the definition of D2.

\emph{Third case : } Suppose $\alpha = i$ for $i: Z \to X$ a
regular closed immersion between smooth schemes. As we have 
already seen, the statement of naturality is local in $X$. 
Therefore, by factoring $i$, we make the reduction to the case
where $Z$ is of codimension 1. We can also assume that $Z$ is a
principal divisor parametrized by $\pi \in \O_X(U)$, where $U =
X - Z$.

By Proposition \ref{prop_2_3}, we are reduced to showing that
the following diagram is commutative:
\[
\begin{diagram}
F_*(X)                    & \rTo{i^*}              & F_*(Z) \\
\dTo                      &                        & \dTo   \\
\assocCM{F_*}(\fields(X)) & \rTo{\Special{v}{\pi}} & \assocCM{F_*}(\fields(Z))
\end{diagram}
\]
Considering the naturality of the structural morphism $\loopCM_*$
of the homotopy module $F_*$, we further reduce the to showing
that the following diagram is commutative:
\[
\begin{diagram}
\phi(M(X)(1)[2]) & \rTo{i^*} &                         &                  &            &                  & \phi(M(Z)(1)[2]) \\
\dTo             &           &                         &                  &            &                  & \dEquals         \\
\phi(M(U)(1)[2]) & \rTo{\nu} & \phi(M(\Gmpt \times U)) & \rTo{\gamma_\pi} & \phi(M(U)) & \rTo{\res{X, Z}} & \phi(M(Z)(1)[2])
\end{diagram}
\]
where $\nu$ is the canonical inclusion, and $\gamma_\pi$ is 
induced by $\pi: U \to \Gmpt$ and $\res{X, Z} = \res{X,Z}' \comp 
\purity{X, Z}^{-1}$ in the notation of point (D4) is the residue
morphism on the level motives. As such, the commutativity of the 
diagram result precisely from Deg08B, 2.6.5.

The map $b: F_* \to \cmcohom^0(\cdot; \assocCM{F_*})$ is therefore
a transfer map. In fact, it is clear that the induced map on the sections
of any function fields is an isomorphism. It follows from Deg08B, 1.4
that $b$ is itself an isomorphism.

Finally, it is straightforward to show that $b$ is compatible with
the structure map of the homotopy modules $F_*$ and 
$\cmcohom^0(\cdot; \assocCM{F_*})$, of the latter, see Section
\ref{subsect_assoc_hm}. Indeed, apply the functoriality of $b$ 
with respect to $j_X: \Gmpt \times X \to \A^1 \times X$ and 
$s_1: X \to \Gmpt \times X$.
\end{proof}

The preceding theorem shows that the category of cycle modules is
symmetric monoidal with the neutral element given by the Milnor 
$K$-theory functor. Furthermore, the tensor product is compatible
with the shifts in grading of the cycle modules --- i.e. functors
$-\{\pm 1\}$ in $\HI_*$.

To all smooth scheme $X$, we can associate the following cycle 
module $\assocCM{[X]}$.

By the preceding theorem, the family of cycle modules 
$\assocCM{[X]}\{n\}$ indexed by smooth scheme $X$ and integer 
$n$ form the generators of the $\CycMod$.

Note that tensor product is characterized by the behavior on 
generators. In particular:
\[
\assocCM{[X]}\{n\} \otimes \assocCM{[Y]}\{m\} =
\assocCM{[X \times Y]}\{n + m\}.
\]
Finally, we give an explicit formula for calculating the cycle 
module associated to a smooth scheme $X$ and an integer $n$. 
For all smooth schemes $X$ and $Y$, consider the group
\[
\pi(X, Y) = \cok (\Cor(\A^1 \times Y, X) 
   \xrightarrow{s_0^* - s_1^*} \Cor(Y, X)).
\]
Note that this group extends to regular schemes essentially of 
finite type over $k$ and that the above groups also commute with
projective limits of schemes (cf. Deg07, 4.1.24). If $E$ is a 
field of functions and $X$ is a smooth projective scheme, 
$\pi(\spec(E), X) = \chow{0}{X_E}$, the Chow group of 0 cycles of
$X$ extended to $E$.

From the preceding remarks, we have the following: for all 
function fields $E$ and all smooth projective schemes $X$,
\[
\assocCM{[X]}_0 . E = \chow{0}{X_E}.
\]
Here $- . -$ represents the intersection theory on cycle modules 
developed in \cite{Ro96}.

Furthermore, for all integers $n > 0$,
\begin{align*}
\assocCM{[X]}_n . E &= \cok\big( \bigoplus_{i = 0}^n 
   \chow{0}{\Gmpt^{n - 1} \times X_E} \to 
      \chow{0}{\Gmpt^n \times X_E}\big) \\
\assocCM{[X]}_{-n} . E &= \ker\big(\pi(\mathbb{G}_{m,E}^n, X)
   \to \bigoplus_{i = 0}^n \pi(\mathbb{G}_{m, E}^{n - 1}, X)\big)
\end{align*}
where the map is induced by the obvious injection of $\Gmpt^i 
\times \{1\} \times \Gmpt^{n - 1 - i} \to \Gmpt^n$.

\section{Functoriality of Gersten Resolution --- Part II}

In this paragraph, we complete the results first promised in 3.2.
Fix a cycle module $M$ and let $F^M$ denote the associated 
homotopy module. We could extend the construction \emph{loc. cit.}
to the case of a $k$-scheme of finite type $X$. In particular, we 
associate to this scheme a complex of sheaves on $X_{\Nis}$
\[
M^*_X: V/X \to \gcmplx{M}(V)
\]
and a sheaf $F^M_X = \sheaf{H}^0(M^*_X)$. The complex $M^*_X$ 
again satisfies the properties of Brown-Gersten but the canonical
map
\[
F^M_X \to M^*_X
\]
need not be an isomorphism. Under this setup, we have the 
following result:
\begin{prop}\label{prop_3_10}
In the setting of the preceding remarks, the isomorphism
\[
\rho_X: \cmcohom^p(X; M) \to H^p(X; F^M)
\]
for a smooth scheme $X$ (cf. Equation \ref{eq_cm_hnis_equiv}) is 
natural with respect to finite correspondences.
\end{prop}
\begin{proof}
The proof follows the principle setup of the proof of Lemma
\ref{lem_cm_hnis_equiv_morph}. Let $f: Y \to X$ be a flat morphism
or a regular closed immersion between schemes of finite type. It 
is clear that the constructions of \emph{loc. cit.} generalize and
allow for the definition of canonical maps $\eta_f$ which fits into
a commutative diagram in the derived category of abelian sheaves on
$X_{\Nis}$
\[
\begin{diagram}
F^M_X & \rTo{\tau_f} & Rf_*F^M_Y \\
\dTo  &              & \dTo      \\
M^*_X & \rTo{\eta_f} & Rf_*M^*_Y.
\end{diagram}
\]
As established elsewhere, if $p: Z \to X$ is a finite map, it 
induces (cf Section \ref{sect_cm_funct}) a map of complexes
\[
p_*: \gcmplx{M}(Z) \to \gcmplx{M}(X)
\]
which is natural in $X$ with respect to \'etale morphisms (cf. 
[Ros96, (4.1)]). We obtain from this the canonical map
\[
tr_p: p_*M^*_Z \to M^*_X
\]
which induces the following commutative diagram in the derived 
category\[
\begin{diagram}
Rp_*(F^M_Z) & \rTo{tr^0_p} & F^M_X \\
\dTo        &              & \dTo  \\
Rp_*M^*_Z   & \rTo{tr_p}   & M^*_X.
\end{diagram}
\]
Resuming the proof of the proposition, it suffices to show that
the naturality of $\rho_X$ for a finite correspondence $\alpha \in 
\Cor(X, Y)$ where $\alpha$ corresponds to an integral closed 
subscheme of $X \times Y$. Following the comments immediately 
after Proposition \ref{prop_2_3}, we consider the morphism
\[
X \stackrel{p}{\leftarrow} Z \stackrel{i}{\rightarrow} 
   Z \times X \times Y \stackrel{q}{\to} Y.
\]
By \emph{loc. cit}, $\alpha^* = p_*i^*q^*$. Applying the 
preceding constructions, we obtain the following commutative
diagram in the derived category of abelian groups:
\[
\begin{diagram}
H^p(Y; F^M_Y) & \rTo{(\tau_q)_*} & H^p(Z \times X \times Y; F^M_Y) & \rTo{(\tau_i)_*} & H^p(Z; F^M_Z) & \rTo{(tr^0_p)_*} & H^p(X; F^M_X) \\
\dTo          &                  & \dTo                            &                  & \dTo          &                  & \dTo          \\
H^p(Y; M^*_Y) & \rTo{(\eta_q)_*} & H^p(Z \times X \times Y; M^*_Z) & \rTo{(\eta_i)_*} & H^p(Z; M^*_Z) & \rTo{(tr_p)_*}   & H^p(X; M^*_X).
\end{diagram}
\]
Verify that the composition of maps in the first row coincides
with $\alpha^*$ and the proposition follows.
\end{proof}

Let $F_*$ be a homotopy module, and $M = \assocCM{F_*}$ its 
generic transform. Consider the isomorphism $b: F_* \to F^M_*$ 
established by Theorem \ref{thm_corresp_cyc_mod_hom_mod}. Via the 
isomorphism in Equation \ref{eq_cm_hnis_equiv}, we obtain the 
following isomorphism
\begin{equation}\label{eq_3_11_a}
\epsilon_X: \hnis^n(X; F_*) \stackrel{b_*}{\to} \hnis^n(X; F^M_*) 
   \stackrel{\rho_X^{-1}}{\to}.
\end{equation}
The preceding proposition has the following immediate corollay:
\begin{cor}
With the isomorphism $\epsilon_X$ defined as above, it is natural
in $X$ with respect to all finite correspondences.
\end{cor}

Consider the homotopy module $\Ox$. By the theorem of 
Suslin-Voevodsky recalled in 1.11, for all function fields $E$, 
$\assocCM{\Ox} \simeq \milK_*(E)$. Moreover, this isomorphism is
compatible with the structure of cycle modules. For the norm map,
this follows from [SV00, 3.4.1]. For the residue map associated to
a function field $E$ with valuation $v$, we reduce to showing that
$\res{v}(\pi) = 1$ for the cycle module $\assocCM{\Ox}$, which 
follows from [Deg07, 2.6.5].

We obtain from this the isomorphism of Bloch\footnote{In effect, 
by the isomorphism that we just explained, the graded sheaf $\Ox$
is the sheaf associated with the unramnified Milnor $K$-theory.} 
for all smooth scheme $X$:
\[
\epsilon_B: \hnis^n(X; (\Ox)^{\otimes n}) \to \cmcohom^n(X; 
\milK_*)_n = \chow{n}{X}.
\]
It follows from the above discussion that this isomorphism is 
compatible with transfers. Recall that for all cycle module $M$, 
there exists a pairing $\milK_* \times M \to M$ in the sense of
[Ros96, 1.2]. By [Ros96, par. 14] This induces a pairing
\[
\chow{n}{X} \otimes \cmcohom^m(X; M)_r \to 
   \cmcohom^{m + n}(X; M)_{r + n}.
\]
Considering a homotopy module $F_*$, we have an (iso)morphism of
homotopy modules $\Ox \otimes F_* \to F_*$. For a smooth scheme 
$X$ with diagonal map $\delta: X \to X \times X$, we get a pairing
\[
H^n(X; \Ox)_n \otimes H^m(X; F_*)_r \to H^{m + n}(X; F_*)_{r + n}
\]
defined via two morphisms: $a: [X] \to \Ox(n)[3n]$ and $b: [X] \to
F_*(r)[2r + m]$ in the composition
\begin{align*}
[X] \stackrel{\delta_*}{\to} [X] \otimes [X] 
   &\xrightarrow{a \otimes b} \Ox \otimes 
   F_*(n + r)[2(n + r) + (n + m)] \\
   &\stackrel{\sim}{\rightarrow} F_*(n + r)[2(n + r) + (n + m)]
\end{align*}
We leave the following compatibility result to the reader
\begin{lem}
With the notations introduced above, the following diagram
is commutative:
\[
\begin{diagram}
H^n(X; \Ox)_n \otimes H^m(X; F_*)_r                & \rTo & H^{n + m}(X; F_*)_{n + r} \\
\dTo{\epsilon_X^B} \otimes \epsilon_X              &      & \dTo{\epsilon_X}          \\
\chow{n}{X} \otimes \cmcohom^m(X; \assocCM{F_*})_r & \rTo & \cmcohom^{m + n}(X; \assocCM{F_*})_{n + r}.
\end{diagram}
\]
\end{lem}
Therefore, the isomophism $\epsilon^B_X$ is compatible with the 
tensor product, and the isomorphism $\epsilon_X$ is compatible 
with the structures of cycle/homotopy modules as described above.

Let $\phi: \DM_{gm}^{op} \to \Ab$ be the realization functor 
associated to $F_*$ (cf. section 1.3). By Proposition 
\ref{prop_3_10}, the functor $\phi$ extends the functor 
$\cmcohom^*(\cdot; \assocCM{F_*})$. Therefore, we extended the 
coefficients of the cohomology canonically to any cycle module via
a triangulated realization functor of $\DM_{gm}$. We again write
\[
\epsilon_X: \phi(M(X)(r)[-2r - n] \to 
   \cmcohom^n(X; \assocCM{F_*})_r
\]
the isomorphism which is obtained from the isomorphism presented 
in Equation \ref{eq_3_11_a}.

Let $f: Y \to X$ be a projective morphism of smooth schemes of 
constant relative dimension $d$. In [Deg08A, 2.7], we constructed
the Gysin map $f^*: M(X)(d)[2d] \to M(Y)$ associated to $f$ in 
$\DM_{gm}$.
