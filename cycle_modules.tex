\section{Cycle Modules}

Let us recall definition and results of cycle modules, first
defined in \cite{Ro96}. In the following, let $S$ be a separated 
scheme of finite type over any base field $k$. Let $\fields(B)$ 
be the class of fields $F$ finitely generated over $k$ such that
$\spec F$ emits a map to $B$.

\begin{definition}
A \emph{cycle premodule} on $\fields(B)$ is an object function
that associated to each $F \in \fields(B)$ a $\Z$-graded abelian
group $M(F) = \prod_i M_i(K)$ with the following data:

\begin{enumerate}
\item[\textbf{D1.}] For each $\phi: F \to E$, there is a degree 0
map $\phi_*: M(F) \to M(E)$ called the \emph{restriction map 
associated to $\phi$}

\item[\textbf{D2.}] For each finite $\phi: F \to E$, there is a 
degree 0 map $\phi^*: M(E) \to M(F)$ called the \emph{corestriction
map associated to $\phi$}

\item[\textbf{D3.}] For each $F$, the group $M(F)$ is equipped
with a left $\milK_*(F)$-module, where $\milK_*(F)$ is the Milnor
$K$-ring of $F$.

\item[\textbf{D4.}] For any valuation $v$ of $F$, there exists 
maps $\res{v}: M(F) \to M(\resf(v))$ and $\special{v}{\idprime{p}}$
called the \emph{residue} and \emph{specialization} respectively, 
where $\resf(v)$ is the residue field of $v$ and $\idprime{p}$ is 
a prime of $v$,
\end{enumerate}

subject to the following conditions:

\begin{enumerate}
\item[\textbf{R1a.}] For each $\phi: F \to E$ and $\psi: E \to L$,
$(\psi \comp \phi)_* = \psi_* \comp \phi_*$

\item[\textbf{R1b.}] For each finite $\phi: F \to E$ and $\psi: E
\to L$, $(\psi \comp \phi)^* = \phi^* \comp \psi^*$

\item[\textbf{R1c.}] For $\phi: F \to E$ and $\psi: E \to L$ with
$\phi$ finite, define $R = E \otimes_F L$, and let $\idprime{p}$
be any prime ideal of $R$. (As $R$ is Artin, let $l_p$ be the 
length of the localized ring $R_{(\idprime{p})}$), and 
$\phi_{\idprime{p}}: L \to R/{\idprime{p}}$ and 
$\psi_{\idprime{p}}: E \to R/{\idprime{p}}$ be natural maps.
\[
\psi_*\comp \phi^* = \sum_{\idprime{p}} l_p \cdot 
(\phi_{\idprime{p}})^* \comp (\psi_{\idprime{p}})_*.
\]

\item[\textbf{R2.}] For $\phi: F \to E$, $x \in \milK_*F$, $y \in
\milK_*E$, $\rho \in M(F)$, and $\mu \in M(E)$, then:

\item[\textbf{R2a.}] $\phi_*(x \cdot \rho) = \phi_*(x) \cdot 
\phi_*(\rho)$.

\item[\textbf{R2b.}] if $\phi$ is finite, $\phi^*(\phi_*(x) \cdot 
\mu) = x \cdot \phi^*(\mu)$.

\item[\textbf{R2c.}] if $\phi$ is finite, $\phi^*(y \cdot 
\phi_*(\rho)) = \phi^*(y) \cdot \rho$.

\item[\textbf{R3.}] For $\phi: F \to E$, $v$ a valuation on $E$
and $w$ and a valuation on $F$:

\item[\textbf{R3a.}] Suppose $w$ is a nontrivial restriction of 
$v$ with ramification index $e$. Let $\phib: \resf{w} \to 
\resf{v}$ be the induced map. Then:
\[
\res{v} \comp \phi_* = e \phib_* \comp \res{w}.
\]

\item[\textbf{R3b.}] Let $\phi$ be finite. For each valuation $v$ 
an extension of $w$ to $E$, let $\phi_v: \resf{w} \to \resf{v}$
be the induced map. Then
\[
\res{v} \comp \phi^* \sum_{v} \comp \res{v}.
\]

\item[\textbf{R3c.}] Suppose $v$ restricts to a trivial valuation
$w$ on $F$. Then
\[
\res{v} \comp \phi_* = 0
\]

\item[\textbf{R3d.}] Suppose $v$ again restricts to a trivial 
valuation, and let $\phib: F \to \resf{v}$ be the induced map, and
$\idprime{p}$ a prime of $v$. Then
\[
\special{v}{\idprime{p}} \comp \phi_* = \phib_*
\]

\item[\textbf{R3e.}] Let $u$ be a $v$-unit, and $\rho \in M(F)$,
one has
\[
\res{v}(\{u\} \cdot \rho) = -\{\overline{u}\} \cdot \res{v}(\rho).
\]
\end{enumerate}
\end{definition}

\begin{definition}
A homomorphism $\omega: M \to M'$ of cycle premodules over 
$\fields(B)$ of even (resp. odd) type is given by homomorphisms:
\[
\omega_F: M(F) \to M'(F)
\]
such that for each $\phi: F \to E$
\begin{enumerate}
\item $\phi_* \comp \omega_F = \omega_E \comp \phi_*$

\item $\phi^* \comp \omega_E = \omega_F \comp \phi^*$

\item for $\{a\} \in \milK_*F$ and $\rho \in M(F)$, $\{a\} \cdot 
\omega_F(\rho) = \omega_F(\{a\} \cdot \rho)$ (resp. 
$\{a\} \cdot \omega_F(\rho) = -\omega_F(\{a\} \cdot \rho)$)

\item $\res{v} \comp \omega_F = \omega_{\resf{v}} \comp \res{v}$
(resp. $\res{v} \comp \omega_F = -\omega_{\resf{v}} \comp 
\res{v}$).
\end{enumerate}
\end{definition}

\begin{ex}
$\milK_*$ is a cycle premodule.
\end{ex}

For $X$ a $k$-scheme, let $\subsch{1}{X}$ denote the collection of 
codimension 1 subscheme. Write $\xi_X$ be the generic point of an
irreducible $X$ with $K_X = \O_X,\xi_X$. If $X$ is normal, then 
for $x \in \codim{1}{X}$, the local ring $\O_X,x$ is a valuation 
ring of $K_X$ with residue field $\resf{x}$. Write $M(x)$ for 
$M(\resf{x})$, and $\res{x}: M(\xi_X) \to M(x)$ for the 
restriction map.

Furthermore, for $x, y \in X$, let $Z$ be the closed subscheme 
determined by $x$, and $\overline{Z}$ be the normalization $Z$.
Define
\[
\ptres{x}{y}: M(x) \to M(y)
\]
by
\[
\ptres{x}{y} = 
\begin{cases}
0 & y \notin \subsch{1}{Z} \\
\sum_{z|y} \phi_{\resf{z},\resf{x}}^* \comp \res{z} & \textrm{otherwise}
\end{cases}
\]
In case $\ptres{x}{y}$ is nonzero, the sum is taken over all $z$
lying over $y \in \subsch{1}{Z}$, and
$\phi_{\resf{z},\resf{y}}^*$ is the corestriction map associated to 
the finite field extension $\resf{y} \to \resf{z}$.

\begin{definition}
A cycle module $M$ on $\fields(B)$ is a cycle premodule which
satisfies the following conditions:

\begin{enumerate}
\item[\textbf{(FD)}] \itemhead{Finite support of divisors.} 
$X$ be a normal scheme and $\rho \in M(\xi_X)$. Then $\res{x}: 
M(\xi_X) \to M(X)$ is 0 for all but finitely many $x \in 
\subsch{1}{X}$.

\item[\textbf{(C)}] \itemhead{Closedness.} If $X$ is an integral
local local of dimension 2 with closed point $x_0$, then the map 
from $M(\xi_X)$ to $M(x_0)$ given by
\[
\sum_{x \in \subsch{1}{X}} \ptres{x_0}{x} \comp \ptres{x}{\xi}
\]
is 0.
\end{enumerate}
\end{definition}

Let $\Cat{MCyc}$ denote the category of cycle modules. In ---,
D\'eglise demonstrated that $\Cat{MCyc}$ is categorically 
equivalent to the category of homotopic modules (see -- ), which
we represent by $\HI_*$. 

\vskip 10pt
The complex \[\gersseq{M}\]

We write $C_M^*$ for the Gersten complex associated to the cycle
module $M$. Rost showed that associated to every flat morphism 
$Y \to X$, there exists a contravariant morphism $C_M^*(X) \to 
C_M^*(Y)$, and to every equidimensional proper morphism $Y \to X$, 
a covariant $C_M^*(Y) \to C_M^*(X)$.

In [Deg06, 3.18], Deglise extended Rost's original work and 
associated to all local complete intersections $f: Y \to X$ for
which $Y$ emits resolution of singularities(see Deg06 3.13) a 
\emph{Gysin Map} 
\[
   f^*: C_M^*(X) \to C_M^*(Y)
\]
which is a map in the derived category of abelian groups, given 
by the composition of a morphism with the formal inverse of a 
quasi-isomorphism. More precisely, the latter is a formal inverse 
of the morphism $p^*$ for $p$ the projection map of a vector 
bundle. 

The Gysin map satisfies the following properties:
\begin{enumerate}
\item Since $f$ is also flat, $f^*$ coincide with the flat 
pullback map described earlier.

\item If $g: Z \to Y$ is a local complete intersection with $Z$
emitting resolution of singularities, then $(fg)^* = g^*f^*$.
\end{enumerate}

In the case where $f$ is a regular closed immersion, the 
hypothesis that $Y$ emits resolution of singularities is 
unnecessary. The map $f^*$ can be defined by using deformation
of the normal cone, following the original idea of Rost (cf 
[Deg.06, 3.3]).

We will use the following result due to Rost ([Ros96, 12.4]),
which partially describes the Gysin morphism:

\begin{prop}
Let $X$ be an integrl scheme with function field $E$, and let $i: 
Z \to X$ be the closed immersion of a regular irreducible 
principal divisor, parametrized by $\idprime{p} \in \O_X(X)$. Let 
$v$ be the valuation of $E$ corresponding to the divisor $Z$. 
Therefore, the morphism $i^*: \cmcohom^0(X; M) \to \cmcohom^0(Z; 
M)$ is the restriction of $\special{v}{\idprime{p}}: M(E) \to 
M(\resf{v})$.
\end{prop}

To each cartesian square
\[
\begin{diagram}
Y'      & \rTo{j} & X'      \\
\dTo{g} &         & \dTo{f} \\
Y       & \rTo{i} & X
\end{diagram}
\]
such that $i$ is a regular closed immersion, we associate a 
\emph{refined Gysin morphism} $\refgysin$ in the derived 
category:
\[
\refgysin: C_M^*(X') \to C_M^*(Y')
\]
that satisfies the following properties:
\begin{enumerate}
\item If $j$ is regular and the morphism of the normal cones
$N_{Y'}(X') \to g^{-1}N_Y(X)$ is an isomorphism, then $\refgysin = 
j^*$

\item If $f$ is proper, then $i^*f_* = g_*\refgysin$.
\end{enumerate}

Further, if the canonical immersion of the cone $C_{Y'}(X') \to
g^{-1}N_Y(X)$ in the normal fiber of $i$ is of pure codimension
equal to $e$, then the morphism $\refgysin$ is of cohomological
degree $e$.
