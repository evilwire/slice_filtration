%\documentclass[11pt]{amsart}

%Diagram package
%\usepackage[dvips,midshaft,nohug,silent,heads=littlevee]{diagrams}

% Taylor diagram arrows
%\newarrow{Mapsto}{mapsto}--->
%\newarrow{Equals}=====
%\newarrow{Dashto}{}{dash}{}{dash}>

%\usepackage{amsmath}
%\usepackage{amsfonts}
%\usepackage{amssymb}
%\usepackage{tikz}

% The following causes equations to be numbered within sections

\newcommand{\D}{\mathbf{D}}
\newcommand{\DM}{\mathbf{DM}^{\mathrm{eff}}}
\newcommand{\DMm}{\mathbf{DM}^{\mathrm{eff},-}}
\newcommand{\A}{\mathbb{A}^1}
\newcommand{\Gm}{\mathbb{A}^1 \setminus \{0\}}
%\newcommand{\HI}{HI}
\newcommand{\tDM}{\otimes_{L}^{tr}}
\newcommand{\homDM}{\mathrm{Hom}_{\DMm}}
\newcommand{\rhomDM}{\underline{RHom}_{\DMm}}
\newcommand{\rhom}{\underline{RHom}_{\D^-}}
\renewcommand{\H}{\mathbf{H}}
\newcommand{\tHI}{\otimes^{\mathrm{Htr}}}
\newcommand{\homHI}{\mathrm{Hom}_{\HI}}
\newcommand{\ihomHI}{\underline{Hom}_{\HI}}
\newcommand{\Ox}{\mathcal{O}^{\times}}
%\newcommand{\Z}{\mathbb{Z}}
\newcommand{\slice}[1]{\nu^{#1}}
\newcommand{\anis}{a_{\mathrm{Nis}}}
\newcommand{\sliceHI}{\sigma}
\newcommand{\SliceHI}{\Sigma}

\section{Slice Filtration in $\HI$}

\noindent In the following, we restrict our focus to the category 
$Cor_k$ of correspondences over a perfect field $k$, and we assume 
all sheaves and presheaves are defined on the Nisnevich site of 
$Cor_k$.

Let $\D^-$ be the derived category of bounded above chain 
complexes of Nisnevich sheaves with transfers. Let $\DMm$ be
the category of effective motives obtained from $\D^-$ by 
inverting $\A$-weak equivalences (see \cite{MVW}) By \emph{loc.\ cit.}, $\DMm$ is 
equivalent to the subcategory $\D_{\HI}^-$ of $\D^-$ consisting of 
bounded above $\A$-local complexes. To emphasize the connection
between $\D_{\HI}^-$ and $\DMm$, we will represent objects of
$\DMm$ as a (co)chain complex of sheaves, e.g. $F^*$.

The category $\DMm$ is closed monoidal; it inherits the tensor
product and internal hom from $\D^-$, which we represent by
$\tDM$ and $\rhomDM$ respectively. Furthermore, we write $F^*(n)$ for
$F^* \tDM \Z(n)$.

Furthermore, $\DMm$ is equipped with a $t$-structure in the sense 
of \cite{BBD} induced by the $t$-structure on $\D_{\HI}^-$, and 
we identify the heart of the structure with the category $\HI$ of 
homotopy invariant sheaves with transfers. In particular, we may 
regard $\HI$ as a subcategory of $\DMm$, corresponding to the 
objects of $\D_{\HI}^-$ concentrated in degree 0. We write $F 
\tDM G$, $\homDM(F, G)$ and $\rhomDM(F, G)$ when $F, G \in \HI$ 
for the tensor, hom, and internal hom applied to the 
corresponding objects in $\DM$. Let $\H^0 : \DMm \rightarrow \HI$ 
be the right adjoint to the inclusion of $\HI$ in $\DMm$.

The category $\HI$ is itself a closed monoidal category. We write 
we write $\tHI$ and $\ihomHI$ for the tensor and internal hom 
bifunctors respectively. These functors are related to the tensor 
and internal hom of $\DMm$ via $\H^0$. Namely, for $F, G \in \HI$,
\[
F \tHI G = \H^0(F \tDM G)
\] 
and 
\[
\ihomHI(F, G) = \H^0\rhomDM(F, G).
\]

Recall from \cite{HuKa} Section 1 that there is a slice 
filtration structure on $\DMm$. Writing $\slice{\geq k}\DMm$ for 
the subcategory of $\DMm$ consisting of $F^* \tDM \Z(k)$, the 
slice filtration is given by the following tower of subcategories: 
\[
\DMm = \slice{\geq 0}\DMm \supset \slice{\geq 1}\DMm
\supset \slice{\geq 2}\DMm \supset \cdots
\]
together with reflection functors $\slice{\geq n}: \DMm 
\longrightarrow \slice{\geq n}\DMm$ given by $F^* \mapsto 
\rhomDM(\Z(n), F^*)(n)$. It seems natural to ask: can we obtain 
the slice filtration structure on $\HI$ by applying $\H^0$ to 
$\slice{\geq n} \DMm$? 

The answer is ``yes''.

To prove this, we will define the counterpart of $\slice{\geq n}$ 
in $\HI$. This is obtained from functors $-\tHI\Ox$ and 
$\ihomHI(\Ox, -)$, which are themselves the counterparts in
$\HI$ of the functors $-\tDM\Z(1)$ and $\rhomDM(\Z(1), -)$ 
respectively. As $\Ox \simeq Z(1)[1]$ in $\DMm$, the approach 
here is to verify that $-\tDM\Z(1)$ and $\rhomDM(\Z(1), -)$ 
behave well with respect to $\H^0$, assertions which we will make 
precise presently.

First, since $\Ox \simeq \Z(1)[1]$ in $\DMm$, notice that $-\tHI \Ox$
is by definition $\H^0(-\tDM \Z(1)[1])$. 

Next, for $F \in \HI$, write $F_{-1}$ for the contraction of $F$,
which is a sheaf that sends $X$ to $F(X \times (\Gm))/F(X)$
(see \cite{MVW} 23.5). 

\begin{prop}\label{homhicontract}
Let $F \in \HI$. Then $\ihomHI(\Ox, F) = F_{-1}$.
\end{prop}

For $F \in \HI$, the proposition above can be rewritten as 
\[
\H^0\rhomDM(\Z(1)[1], F) = F_{-1},
\] 
which is a statement about the 0-th cohomology sheaf 
of the corresponding $\A$-local complex, $\rhom(\Z(1)[1], F)$.
From this perspective, Proposition \ref{homhicontract} is a
consequence of the following more general result:

\begin{lem}\label{homhi}
For any $F^* \in \DMm$,
\[
\H^0\rhomDM(\Z(1)[1], F^*) = \rhomDM(\Z(1)[1],  \H^0F^*).
\]
In particular, for a homotopy invariant sheaf $F$, 
we have 
\[
\rhomDM(\Z(1)[1], F) = F_{-1}.
\]
\end{lem}

\begin{proof}
(D\'eglise) \cite{Deg08b} Proposition 3.4.5.
\end{proof}


\begin{rmk}
It makes sense to regard $F_{-1}$ as an object of $\DMm$,
because, as inferred from above (and can be proven directly),
the contraction $F_{-1}$ of a homotopy invariant sheaf $F$ is 
homotopy invariant. 
\end{rmk}

From these observations, we have the following:

\begin{cor}\label{hiadj}
The functors $ - \tHI \Ox$ is left adjoint to $\ihomHI(\Ox, -)$.
\end{cor}
\begin{proof}
Fix $F, G \in \HI$. By Proposition \ref{homhicontract}, it 
suffices to show that $\homHI(F \tHI \Ox, G) \simeq \homHI(F, 
\ihomHI(\Ox, G)).$ 

By definition $F \tHI \Ox = \H^0(F(1)[1])$ and therefore
\[
\homHI(F \tHI \Ox, G) = \homHI(\H^0(F(1)[1]), G).
\]
Since $G$ is in the heart of $\DMm$, we have that
\[
\homHI(\H^0(F(1)[1]), G) = \homDM(F(1)[1], G).
\]
By adjointness of $-\tDM \Z(1)[1]$ and $\rhomDM(\Z(1)[1], -)$,
\[
\homDM(F(1)[1], G) = \homDM(F, \rhomDM(\Z(1)[1], G)),
\]
and, by Lemma \ref{homhi}, we have that
\[
\homDM(F, \rhomDM(\Z(1)[1], G)) = \homDM(F, G_{-1}).
\]
Since both $F$ and $G_{-1}$ are in the heart of $\DMm$,
\[
\homDM(F, G_{-1}) = \homHI(F, G_{-1}),
\]
and the result is established.
\end{proof}

\begin{cor}
For $F \in \HI$, the adjunction map 
\[
F \longrightarrow \ihomHI(\Ox, F \tHI \Ox)
\] 
is a natural isomorphism.
\end{cor}
\begin{proof}
By the Cancellation Theorem for $\DMm$ (\cite{MVW} 16.25), the 
adjunction map 
\[
F^* \longrightarrow \rhomDM(\Z(1)[1], F^*(1)[1]).
\] 
is a natural isomorphism. In particular, the map 
\[
F \longrightarrow \rhomDM(\Z(1)[1], F(1)[1])
\]
is an isomorphism for any $F \in \HI$. Consequently, the sheaf 
$\H^0 F$ is isomorphic to $\H^0\rhomDM(\Z(1)[1],\allowbreak 
F(1)[1]).$ However, by Lemma \ref{homhi} the latter is isomorphic 
to $\ihomHI(\Ox,\allowbreak F \tHI\Ox)$. The corollary now 
follows from the fact that $F = \H^0 F$.
\end{proof}

We are now ready to construct the slice filtration on $\HI$. 
To simplify notations, in the following, for a homotopy invariant
sheaf $F$, let $L(F)$ denote $F \tHI \Ox$, and $R$ denote 
$\ihomHI(\Ox, F) = F_{-1}$; for any positive integer $n$, we 
write $L^n$ and $R^n$ for the $n$-th iterations of $L$ and $R$ 
respectively.

Let $\SliceHI_{< n} \HI$ denote the full subcategory of $\HI$ 
consisting of those $F$ where $R^n(F) = 0$. It is clear that we 
have an ascending tower of subcategories:
\[
\SliceHI_{0} \HI \subset \cdots \SliceHI_{< n - 1} \HI \subset 
\SliceHI_{< n} \HI \subset \cdots \HI.
\]
Furthermore, as $(L, R)$ is an adjoint pair, there exists a 
natural map from $L^nR^n(F)$ to $F$. Let $\sliceHI_{< n}(F)$ represent 
the cokernel of this map. The association $F \mapsto 
\sliceHI_{< n}(F)$ is functorial. In fact,

\begin{thm}\label{slicethm}
The functor $\sliceHI_{< n}$ is left adjoint to the inclusion 
of $\SliceHI_{< n} \HI$ into $\HI$.
\end{thm}

The proof of the theorem depends on the following:

\begin{lem}\label{hislice}
For $F \in \HI$, $\sliceHI_{< n}(F) \in \SliceHI_{< n}\HI$. 
Furthermore, if $F \in \SliceHI_{< n} \HI$, $\sliceHI_{<n}(F)
\simeq F$.
\end{lem}
\begin{proof}
By definition, $\sliceHI_{<n}(F) \in \SliceHI_{< n}\HI$ if
and only if $R^n \sliceHI_{<n}(F) = 0$.

Now, by \cite{Deg10} Lemma 1.14 the functor $R$ is exact. Applying 
$R^n$ to the exact sequence
\begin{equation}\label{sliceExSeq}
L^nR^n F \longrightarrow F \longrightarrow \sliceHI_{< n}(F)
\longrightarrow 0,
\end{equation}
we again have an exact sequence:
\[
R^nL^nR^n (F) \longrightarrow R^n(F) \longrightarrow 
R^n\sliceHI_{< n}(F) \longrightarrow 0.
\]
But the map $R^nL^nR^n(F) \rightarrow R^n(F)$ is an isomorphism 
by Lemma \ref{hiadj}. Hence, $R^n\sliceHI_{< n}(F) = 0$.

For the second assertion, if $F \in \SliceHI_{< n} \HI$,
by definition we have $L^nR^nF = 0$. The exact sequence 
(\ref{sliceExSeq}) gives us the desired isomorphism.
\end{proof}

We now prove Theorem \ref{slicethm}.

\begin{proof}[Proof of Theorem \ref{slicethm}]
Fix $F \in \HI$ and $G \in \SliceHI_{<n}\HI$. Consider
$f: F \to G$. Applying $\sliceHI_{<n}$, we have a map
$\sliceHI_{<n}(f): \sliceHI_{<n}(F) \to \sliceHI_{<n}(G)$.
Since $G \in \SliceHI_{<n}\HI$, by Lemma \ref{hislice},
$\sliceHI_{<n}(G) \simeq G$. Composing with $\sliceHI_{<n}(f)$,
we obtain a map $\sliceHI_{<n}(F) \to G$, which, by abuse of
notation, we write as $\sliceHI_{<n}(f)$.

Let $\phi: \homHI(F, G) \longrightarrow
\mathrm{Hom}_{\SliceHI_{< n}\HI}(\sliceHI_{<n}(F), G)$
represent the map $f \mapsto \sliceHI_{< n}(f)$. To show that 
$\phi$ is injective, notice that we have the following 
commutative diagram:
\begin{equation*}
\begin{diagram}[height=0.8cm]
 F & \rTo^{f}   && G \\ 
      \dTo{}    &&     & \dTo{\simeq}  \\
\sliceHI_{<n}(F) &&\rTo^{\sliceHI_{< n}(f)} & \sliceHI_{<n}(G)
\end{diagram}
\end{equation*}

Since $G \to \sliceHI_{< n}(G)$ is an isomorphism, 
$\sliceHI_{<n}(f) = 0$ implies that $f = 0$.

For surjectivity of $\phi$, fix $g: \sliceHI_{<n}(F) \to G$.
Composing with $F \to \sliceHI_{<n}(F)$ gives the desired
map in $\homHI(F, G)$.

\end{proof}
\bibliographystyle{plain}
\begin{thebibliography}{10}

\bibitem{BBD}
J. Bernstein, A. Beilinson and P. Deligne,
Faisceaux pervers, {\em Asterisque} 100 (1982).

\bibitem{Deg08b}
F. D\'eglise, Motifs generiques.
{\em Rendiconti Sem. Mat. Univ. Padova},
119 (2008), 173 - 244.

\bibitem{Deg10}
F. D\'eglise, 
Modules homotopiques.
{\em Documenta Math.}
16 (2011), 411 - 455

\bibitem{HuKa}
F. Huber, B. Kahn, The slice filtration and mixed Tate motives,
Compositio Math. 142 (2006), 907 - 936.

\bibitem{MVW}
C. Mazza, V. Voevodsky, C. Weibel,
{\em Lecture notes on motivic cohomology},
Clay Mathematics Monographs vol. 2,
AMS, 2006

\bibitem{WH}
C. Weibel,
{\em An introduction to homological algebra},
Cambridge Univ. Press, 1994.

\end{thebibliography}


%Does anything here even show up?

%%%%%%%%%%%%%%%%%%%%%%%%%%%%%%%%%%%%%%%%%%%
%
% TODO:
%
% * fixed \Ox 
% * fixed "denote"
% * obtained from rm:the:rm functors
% * Clarified HuKa source
% * add:assertion:add which we will make precise
% * add:(D\'eglise):add See \ref{{*}} 1.13, and
% * s/Certainly/Consequently/
% * s/a descending/an ascending/
% * provide reference for t-structure
% * provide definition for \H^0
% * provide definition for ihomHI and tHI
% * add:and write F(n) for ...:add
% * fix references
% * prepare proofs for Corollary 1.5
% * turn 1.4 into just a statement
% * fix I the more
% * Furth(s/re/er/)more
% * combine Lemmas 1.8 and 1.9 
% * fix consistency of notation for the nth slice
% * label commutative diagram
% * fix reference for Motifs Generiques
% * fix reference for HuKa
%
%
%%%%%%%%%%%%%%%%%%%%%%%%%%%%%%%%%%%%%%%%%%%
