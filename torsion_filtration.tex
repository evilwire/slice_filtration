\newcommand{\olDM}[1]{\sigma_{\leq #1}}
\newcommand{\ogDM}[1]{\sigma_{\geq #1}}
\newcommand{\nlDM}[1]{\nu_{\leq #1}}
\newcommand{\ngDM}[1]{\nu_{\geq #1}}
\newcommand{\ohomeHI}[0]{\underline{Hom}_{\sigma \geq i\HI}}
\newcommand{\torT}[0]{\mathbb{T}}
\newcommand{\torF}[0]{\mathbb{F}}
\newcommand{\xlongrightarrow}[1]{\stackrel{#1}{\longrightarrow}}
\newcommand{\homTorTn}[0]{\mathrm{Hom}_{\torT_n}}

By --, we have that for each $F$ there exists a sequence of right 
exact sequences of the form 

\begin{equation}
L^nR^n F \longrightarrow F \longrightarrow \olDM{n}(F)
\longrightarrow 0.
\end{equation}

where $\olDM{n}(F) \in \olDM{n} \HI$. Since $H^0 \nlDM{n-1} iF = 
\olDM{n}(F)$, we see that $\olDM{n}\HI=H^0\nlDM{n}\DMm$. 
Similarly, we see that, via $H^0$, the sequence of subcategories

\begin{equation}
\DMm=\ngDM{0}\DMm,\ngDM{1}\DMm,... 
\end{equation}

of $\DMm$ gives rise to a sequence of subcategories

\begin{equation}
\HI=\ogDM{0} \HI, \ogDM{1} \HI, ...
\end{equation}

where $\ngDM{i} \HI$ is the full subcategory of $\HI$ consisting 
of sheaves of the form $F \tHI (\Ox)^{\otimes i}$. Indeed, we have 
the following tower of inclusions

\begin{equation}
\cdots \longrightarrow \ngDM{i} \HI \longrightarrow \ngDM{i-1} \HI 
\longrightarrow \cdots
\end{equation}

Together with reflection functors $\HI \longrightarrow \ogDM{i} 
\HI$ given by $F \mapsto L^iR^i F$. In particular:

\begin{prop}\label{functer_prop}
The functor $L^iR^i$ is right adjoint to the inclusion of $\ogDM{i} 
\HI \longrightarrow \HI$.
\end{prop}

\begin{proof}[Proof of Proposition \ref{functer_prop}]

For $F \in \ogDM{i} \HI$, and $G \in \HI$, fix $\phi \in 
\ihomHI(F, G)$. Then, we have the following commutative diagram:

\begin{equation*}
\begin{diagram}[height=0.8cm]
 L^iR^iF & \rTo^{L^iR^i\phi}   && L^iR^iG \\ 
      \dTo{}    &&     & \dTo{}  \\
F &&\rTo^{\phi} & G
\end{diagram}
\end{equation*}

Note that counit map $L^iR^iF \longrightarrow F$ is an 
isomorphism. This allows us to define a map

\begin{equation}
r: \ihomHI(F,G) \longrightarrow \ohomeHI(FL^iR^iG)
\end{equation}

given by $\phi \mapsto L^iR^i\phi \epsilon_F^{-1}$ if 
$L^iR^i\phi=0$, then $\phi\epsilon_F=\epsilon_GL^iR^i\phi=0$, but 
$\epsilon_G$ is an isomorphism. Thus, $r$ is injective.  

Fix $\psi : F \longrightarrow L^iR^iG$ and set $\phi = \epsilon_g 
\cdot \psi$. We claim that $r(\psi)=L^iR^i\phi\epsilon_F^{-1} = 
\psi$. We have the following:

\begin{equation*}
\begin{diagram}[height=0.8cm]
 F & \rTo^{\psi}   && L^iR^iG \\ 
      \dTo{\epsilon_F}    &&     & \dTo{\epsilon_{L^iR^iG}}  \\
L^iR^iF &&\rTo^{L^iR^i\psi} & (L^iR^i)^2G
\end{diagram}
\end{equation*} 

That is, $\psi \epsilon_F = \epsilon_{L^iR^iG} L^iR^i\psi$. 
Furthermore, we have that $L^iR^i\phi\epsilon_F^{-1} = 
L^iR^i(\epsilon_G \cdot \psi)_{\epsilon_F}^{-1} = 
\epsilon_{L^i R^iG}L^i R^i \psi_{\epsilon_F}^{-1} = 
\psi\epsilon^F\cdot\epsilon_F^{-1}=\psi$ as desired.
\end{proof}

As a consequence of the above, we have that

\begin{cor}
The sequence

\begin{equation}
\HI = \ogDM{0}\HI \longleftarrow \ogDM{1} \HI \longrightarrow 
\cdots
\end{equation}

is a filtration of $\HI$.
\end{cor}

However, $\HI$ does not admit a functional filtration by $\{\nu 
\geq_{i} \HI\}$, since the coreflection functors $F \mapsto 
L^iR^iF$ are not generally injective, as demonstrated by the 
following example: 

\begin{ex}
Let $(\Ox)^n$ be the sheaf associated to the presheaf where 
sections of $X \in \delta m /\epsilon$ is the abelian subgroup of 
$\Ox$ given by $\{x \in \Ox | x=y^n$ for some $y \in \Ox(X)\}$. 
It is clear that $(\Ox)^n$ is a homotopy invariant sheaf with 
transfers. Furthermore, there exists the following exact sequence 

\begin{equation}
0 \longrightarrow \mu_n \longrightarrow \O^x \longrightarrow \O^{xn} 
\longrightarrow 0
\end{equation}

We note that $R(\mu_n) = 0$ (see MVW). Therefore, we have

\begin{equation}
0 \longrightarrow LR\O^x \longrightarrow LR(\Ox)^{n} \longrightarrow 0
\end{equation}

In particular, $LR(\Ox)^{n} \simeq LR(\Ox) \simeq \Ox$, and the counit 
maps $LR\O^{xn} \longrightarrow (\Ox)^n$ is precisely $n$, wihch has 
a nontrivial kernel.

The snag is that the size of $\ogDM{i} \HI$ is simply too small 
and doesn't include the kernals of the map

\begin{equation}
F \longrightarrow \olDM{n} F
\end{equation}

This can be resolved by enlarging the filtration at each level, 
and to do so, we turn to torsion theory of a well-powered abelian 
category.  
\end{ex}

The goal is to construct a torsion theory ($\torT_n$,$\torF_n$) 
for each $n$, and show that there exists a natural inclusion of 
full subcategories $\torT_i \longrightarrow \torT_{i-1}$ that fit 
together in the following tower

\begin{equation}
\cdots \longrightarrow \torT_i \longrightarrow \torT_{i-1} 
\longrightarrow \torT_{i-2} \longrightarrow \cdots 
\longrightarrow \torT_0 = \HI
\end{equation}

and that $\HI$ admits a functional filtration with respect to 
$\{\torT_i\}$.

First, we construct the sequence of torsion theories.

\begin{lem}
The function $\olDM{n}$ is idempotent.
\end{lem}

\begin{proof}
Fix $F \in \HI$. By --, $\olDM{n} F \in \olDM{n} \HI$,
and by definition $R^{n+1} \olDM{n} F = 0$, and, thus,
$L^{n+1}R^{n+1}F=0$. It follows that $\olDM{n} F = 
\olDM{n}\olDM{n}F$.
\end{proof}

\begin{lem}
$\sigma_n(\ker F \longrightarrow \olDM{n} F) = 0$.
\end{lem}

\begin{proof}
Fix $F \in \HI$, let $F \longrightarrow \olDM{n}F$ be the 
canonical surjection, and let $K_n$ be the kernel. Then we have:

\begin{equation*}
\begin{diagram}[height=0.8cm]
 & & L^{n+1}R^{n+1}K & \rTo^{}   & L^{n+1}R^{n+1}F & \rTo & 0       & \rTo & 0 \\ 
 & &     \dTo{}    &            & \dTo{}           &     & \dTo{}\\
0 & \rTo^{} & K_n & \rTo^{} & F & \rTo^{} & \olDM{n}F & \rTo^{} & 0 \\ 
 & &     \dTo{}    &            & \dTo{}           &     & \dTo{}\\
 & &    \olDM{n} K_n & \rTo^{} & \olDM{n}F & \rTo^{} & \olDM{n}^2F & \rTo^{} & 0 \\
\end{diagram}
\end{equation*}

By the Snake Lemma, we have the exact sequence

\begin{equation}
0 \longrightarrow \olDM{n}K_n \longrightarrow \olDM{n} F 
\longrightarrow \olDM{n}^2 F \longrightarrow 0
\end{equation}
\end{proof}

\begin{prop}
For each $n$, $\olDM{n}$ is a coradical.
\end{prop}

\begin{proof}
$\olDM{n}$ is a right exact quotient functor by --. -- shows that 
$\olDM{n}$ is idempotent, and -- shows that $\olDM{n} (\ker F 
\longrightarrow \olDM{n}F) = 0$ for each $F$.
\end{proof}

\begin{cor}
For each $n$, there exists a hereditary torsion theory $(\torT_n, 
\torF_n)$.
\end{cor}

\begin{proof}
This is a direct consequence of Prop -- and Theorem --.
\end{proof}

Let us give a more explicit description of the torsion theories 
$(\torT_n, \torF_n)$. Following the construction, we have

\begin{align*}\label{f0}
F \in \torT_n \textrm{ if and only if } \olDM{n} F = 0. \\
F \in \torF_n \textrm{ if and only if } \olDM{n} F = F.
\end{align*}

Tying this together with the filtration we have introduced 
earlier, we have

\begin{prop}
$\torF_n = \olDM{n} \HI$, and $\torT_n = \{F \in \HI | L^nR^nF 
\longrightarrow F$ is a surjection $\}$.
\end{prop}

\begin{proof}
If $\olDM{n}F = F$. Then $R^{n+1}F=R^{n+1}\olDM{n}F=0$. Therefore 
$F \in \olDM{n} \HI$. Conversely, fix $F \in \olDM{n} \HI$, we have

\begin{equation}
0 = L^{n+1}R^{n+1}F \longrightarrow F \longrightarrow \olDM{n}F 
\longrightarrow 0.
\end{equation}

It follows that $F= \olDM{n}F$.

That $L^{n+1}R^{n+1}F \longrightarrow F$ is a surjection iff and 
only if $\olDM{n}F = 0$. The second statement follows.
\end{proof}

Since $L^nR^n(L^{n+k}R^{n+k}F) \stackrel{\sim}{\longrightarrow}
L^{n+k}R^{n+k} F$ is an isomorphism for all $k > 0$ (in particular 
a surjection), we have that $\torT_{n+k} \subset \torT{n}$. 
Furthermore, for $F \in \HI$, for $F \longrightarrow \olDM{n} F$, 
complete to an exact sequence:

\begin{equation}
0 \longrightarrow K_n \xlongrightarrow{iF} F 
\xlongrightarrow{S_I} \olDM{n}F \longrightarrow 0
\end{equation}

We claim that sending $F$ to $K_n$ defines a functor $\HI 
\longrightarrow \torT_n$. Indeed, fix $F \xlongrightarrow{\phi} 
G$, and we have

\begin{equation*}
\begin{diagram}[height=0.8cm]
0 & \rTo^{}   & K_n & \rTo^{i_F} & F       & \rTo & \olDM{n} F & \rTo & 0 \\ 
     \dTo{}    &            & \dTo{\phi}           &     & \dTo{}\\
0 & \rTo^{} & K_n & \rTo^{} & G & \rTo_{S_g} & \olDM{n}G & \rTo & 0 \\ 
\end{diagram}
\end{equation*}
and since $S_g\phi i _F = 0$, we have a map $K_n\longrightarrow 
K_n'$. It is straightforward to verify that this association 
preserves composition. Let $K_n$ denote this functor.
 
\begin{lem}
$K_n$ is idempotent. In particular, $K_n|_{\torT_n}$ is the identity.
\end{lem}

\begin{proof}
Since $\olDM{n-1}K_n(F)=0$, we have

\begin{equation}
0 \longrightarrow K_n^2(F) \longrightarrow K_n(F) \longrightarrow 
\olDM{n-1} K_n F = 0.
\end{equation}
\end{proof}

\begin{lem}
$K_n$ is right adjoint to inclusion $\torT_n\longrightarrow\HI$.
\end{lem}

\begin{proof}
Fix $F \in \torT_n$ and $G \in \HI$. Then we have for each 
$\phi: F \longrightarrow G$.

\begin{equation*}
\begin{diagram}[height=0.8cm]
 K_nF & \rTo^{K_n\phi}   && K_nG \\ 
      \dTo{}    &&     & \dTo{i_g}  \\
F &&\rTo^{\phi} & G
\end{diagram}
\end{equation*}

and definte $r: \homHI(F,G) \longrightarrow \homTorTn(K_nF,G)$ by 
$\phi \in \homHI(F,G) \mapsto K_n\phi$, with inverse given by 
$\psi \in \homTorTn(F,G) \mapsto i_g\psi$.
\end{proof}

\begin{thm}
The tower

\begin{equation}
\cdots \longrightarrow \torT_i \longrightarrow \torT_{i-1} 
\longrightarrow \cdots \longrightarrow \torT_0 = \HI
\end{equation}

is a filtration of $\HI$, which equips $\HI$ with an associated 
functorial filtration by $\left\{ \torT_i \right\}$.
\end{thm}

\begin{ex}
For $F \in \ogDM{n} \HI$, $F=L^{i+1}R^{i+1}F'$ for some $F' \in 
\HI$. Therefore 
\begin{align*}
L^{i+1}F^{i+1}F &= L^{i+1}R^{i+1}L^{i+1}R^{i+1}F' \\
&= L^{i+1}R^{i+1}F' \\
&= F
\end{align*}

Then $\ogDM{n} \HI$ forms a full subcateogry of $\torT_n$.
\end{ex}

\begin{ex}
The sheaf $(\Ox)^n$ in example -- is an object of $\torT_1$
but is not an object of $\ogDM{1} \HI$.
\end{ex}

To wrap up this section, we have the following properties of the 
functors $K_n$ and $\olDM{n}$ in preparation for section --.

\begin{prop}
Both $K_n$ and $\olDM{n}$ are exact.
\end{prop}

\begin{proof}
Since $K_n$ is a right adjoint, and $\olDM{n}$ is a left adjoint, 
it suffices to verify that $K_n$ is right exact and $\olDM{n}$ is 
left exact.

To see this, consider a sequence:

\begin{equation}
0 \longrightarrow F' \longrightarrow F \longrightarrow F'' \longrightarrow 0
\end{equation}

and we have 

\begin{equation*}
\begin{diagram}[height=0.8cm]
 0 & \rTo & K_nF' & \rTo^{}   & K_nF & \rTo & K_nF''       & \\ 
  & &    \dTo{}    &            & \dTo{}           &     & \dTo{}\\
0 & \rTo^{} & F' & \rTo^{} & F & \rTo^{} & F'' & \rTo^{} & 0 \\ 
 & &     \dTo{}    &            & \dTo{}           &     & \dTo{}\\
 & &    \olDM{n}F' & \rTo^{} & \olDM{n}F & \rTo^{} & \olDM{n}F'' & \rTo^{} & 0 \\
  & &     \dTo{}    &            & \dTo{}           &     & \dTo{}\\
 && 0 && 0 && 0
\end{diagram}
\end{equation*}

In fact, it suffices to show that $K_n$ is right exact (and Snake 
lemma will force $\olDM{n}$ to be left exact.) The fact that 
$\olDM{n}F' \longrightarrow \olDM{n}F \longrightarrow \olDM{n}F^n 
\longrightarrow 0$ is exact implies that $\olDM{n}F'$ sujects 
onto $\ker(\olDM{n}F \longrightarrow \olDM{n}F''$. Therefore, by 
the Snake Lemma, $K_nF \longrightarrow K_nF''$ is surjective.  
\end{proof}
